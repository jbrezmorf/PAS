
% $Header: /cvsroot/latex-beamer/latex-beamer/solutions/generic-talks/generic-ornate-15min-45min.en.tex,v 1.5 2007/01/28 20:48:23 tantau Exp $

\documentclass[smaller]{beamer}
\mode<presentation>
{
  \usetheme{Singapore}
  \usefonttheme[onlymath]{serif}
  % or ...
 %  \setbeamercovered{transparent}
  % or whatever (possibly just delete it)
}


\usepackage[czech]{babel}
% or whatever
\usepackage[utf8]{inputenc}
% or whatever
%\usepackage{times}
%\usepackage[T1]{fontenc}
% Or whatever. Note that the encoding and the font should match. If T1
% does not look nice, try deleting the line with the fontenc.


\title{PAS02 - basics of probability theory}

\author{Jan B\v rezina}
\institute % (optional, but mostly needed)
{
  %\inst{2}%
  Technical University of Liberec
}


% If you wish to uncover everything in a step-wise fashion, uncomment
% the following command: 

%\beamerdefaultoverlayspecification{<+->}

% ***************************************** SYMBOLS
\def\div{{\rm div}}
\def\Lapl{\Delta}
\def\grad{\nabla}
\def\supp{{\rm supp}}
\def\dist{{\rm dist}}
%\def\chset{\mathbbm{1}}
\def\chset{1}

\def\Tr{{\rm Tr}}
\def\sgn{{\rm sgn}}
\def\to{\rightarrow}
\def\weakto{\rightharpoonup}
\def\imbed{\hookrightarrow}
\def\cimbed{\subset\subset}
\def\range{{\mathcal R}}
\def\leprox{\lesssim}
\def\argdot{{\hspace{0.18em}\cdot\hspace{0.18em}}}
\def\Distr{{\mathcal D}}
\def\calK{{\mathcal K}}
\def\FromTo{|\rightarrow}
\def\convol{\star}
\def\impl{\Rightarrow}
\DeclareMathOperator*{\esslim}{esslim}
\DeclareMathOperator*{\esssup}{ess\,sup}
\DeclareMathOperator{\ess}{ess}
\DeclareMathOperator{\osc}{osc}
\DeclareMathOperator{\curl}{curl}

%\def\Ess{{\rm ess}}
%\def\Exp{{\rm exp}}
%\def\Implies{\Longrightarrow}
%\def\Equiv{\Longleftrightarrow}
% ****************************************** GENERAL MATH NOTATION
\def\Real{{\rm\bf R}}
\def\Rd{{{\rm\bf R}^{\rm 3}}}
\def\RN{{{\rm\bf R}^N}}
\def\D{{\mathbb D}}
\def\Nnum{{\mathbb N}}
\def\Measures{{\mathcal M}}
\def\d{\,{\rm d}}               % differential
\def\sdodt{\genfrac{}{}{}{1}{\rm d}{{\rm d}t}}
\def\dodt{\genfrac{}{}{}{}{\rm d}{{\rm d}t}}

\def\vc#1{\mathbf{\boldsymbol{#1}}}     % vector
\def\tn#1{{\mathbb{#1}}}    % tensor
\def\abs#1{\lvert#1\rvert}
\def\Abs#1{\bigl\lvert#1\bigr\rvert}
\def\bigabs#1{\bigl\lvert#1\bigr\rvert}
\def\Bigabs#1{\Big\lvert#1\Big\rvert}
\def\ABS#1{\left\lvert#1\right\rvert}
\def\norm#1{\bigl\Vert#1\bigr\Vert} %norm
\def\close#1{\overline{#1}}
\def\inter#1{#1^\circ}
\def\ol#1{\overline{#1}}
\def\ul#1{\underline{#1}}
\def\eqdef{\mathrel{\mathop:}=}     % defining equivalence
\def\where{\,|\,}                    % "where" separator in set's defs
\def\timeD#1{\dot{\overline{{#1}}}}

% ******************************************* USEFULL MACROS
\def\RomanEnum{\renewcommand{\labelenumi}{\rm (\roman{enumi})}}   % enumerate by roman numbers
\def\rf#1{(\ref{#1})}                                             % ref. shortcut
\def\prtl{\partial}                                        % partial deriv.
\def\Names#1{{\scshape #1}}
\def\rem#1{{\parskip=0cm\par!! {\sl\small #1} !!}}

\def\Xint#1{\mathchoice
{\XXint\displaystyle\textstyle{#1}}%
{\XXint\textstyle\scriptstyle{#1}}%
{\XXint\scriptstyle\scriptscriptstyle{#1}}%
{\XXint\scriptscriptstyle\scriptscriptstyle{#1}}%
\!\int}
\def\XXint#1#2#3{{\setbox0=\hbox{$#1{#2#3}{\int}$}
\vcenter{\hbox{$#2#3$}}\kern-.5\wd0}}
\def\ddashint{\Xint=}
\def\dashint{\Xint-}

% ******************************************* DOCUMENT NOTATIONS
% document specific
\def\rh{\varrho}
\def\vl{{\vc{u}}}
\def\th{\vartheta}
\def\vx{\vc{x}}
\def\vX{\vc{X}}
\def\vr{\vc{r}}
\def\veta{\vc{\eta}}
\def\dx{\,\d\vx}
\def\dt{\,\d t}
\def\bulk{\zeta}
\def\cS{\close{S}}
\def\eps{\varepsilon}
\def\phi{\varphi}
\def\Bog{{\mathcal B}}
\def\Riesz{{\mathcal R}}
\def\distr{\mathcal D}
\def\Item{$\bullet$}

\def\MEtst{\mathcal T}
%***************************************************************************
% highlight color
\setbeamercolor{my blue}{fg=blue}
\def\blue#1{{\usebeamercolor[fg]{my blue} #1}}

\setbeamercolor{my green}{fg=green}
\def\green#1{{\usebeamercolor[fg]{my green} #1}}

% color for term definition
\setbeamercolor{my orange}{fg=orange}
\def\df#1{{\usebeamercolor[fg]{my orange} #1}}
\def\xskip{{\vspace{2ex}}}

\def\cz#1{{\small (#1)}}

\begin{document}

\begin{frame}
  \titlepage
\end{frame}


\begin{frame}{What is the probability?}
\pause
"What is the probability of passing the exam?" How you define it? \pause 
(empirical - relative frequency)\\

\xskip
What is the probabilty to guess 6 numbers from 40? \pause \\
(model - classical prob.; $ \binom{40}{6} = \frac{40!}{34!6!}~\dots$)\\

\xskip
Probability to hit center in darts. What is center?\pause \\
\dots density of probability, probability as a \df{set function}

\end{frame}

\begin{frame}{Sample space}
\df{Random experiment}\cz{náhodný pokus} produce:\\
\df{outcome}\cz{výsledek} -- not determined, repeatable

\xskip
\df{Sample space}\cz{prostor elementárních jevů} $\Omega$ -- set of all possible outcomes,
range\cz{obor hodnot} of random experiment\\


\xskip
\blue{Examples:}\\

\blue{Discrete outcomes/variables}

taking socks from a bin: $\Omega=\{white, gray, black\}$, \\
throwing die: $\Omega = \{1,2,3,4,5,6\}$,\\

\blue{Continuous outcomes/variables}

measurement of temperature: $\Omega = (0,\infty)$,\\
measurement of a velocity vector: $\Omega= \Real^3$,\\

\blue{"Theoretical" outcome}

motor vibrations during one second: $\Omega=C([0,1])$\\
\dots set of continuous functions on interval $[0,1]$, 
\end{frame}

\begin{frame}{Random event}
\df{random event}\cz{náhodný jev} -- property or statement about outcome\\
formally: subset $\omega$ of $\Omega$\\

\blue{Examples}\\
\begin{itemize}
\item will not draw gray sock; will draw white or black,\\
      $\omega = \{white, black\}$\\
\item will roll odd number with dice,\\
      $\omega = \{1,3,5\}$\\
\item temperature $[K]$ allowing liquid water at see pressure,\\
      $\omega=(273,373)$, \\
\item wind will blow in north-east direction,\\
      $\omega=\{a,a\},\quad a>0$
\end{itemize}
\xskip
\df{elementary events}\cz{elementární jevy} -- possible outcomes, elements of $\Omega$
\end{frame}


\begin{frame}{Relations and operations with events=sets}
\blue{relations between events:}
\begin{align*}
 A\subset B&\quad\text{event $A$ implies event B}\\
 A = B  &\quad\text{$A$ comes when ever $B$ comes}\\
 A\cap B =\emptyset&\quad\text{disjoint events}
\end{align*} 

\blue{operations with events:}
\begin{align*}
 A\cup B  &\quad\text{event $A$ or event $B$ }\\
 A\cap B  &\quad\text{event $A$ and event $B$}\\
 \overline{A}, A^c &\quad\text{$A$ will not come},\text{ \df{complementary event} (doplňkový jev)}\\
 A\setminus B &\quad\text{$A$ will come, but not $B$}
\end{align*}
also countable unions and intersections:
\[
  \bigcup_{i\in N} A_i,\quad \bigcap_{i\in N} A_i
\]
\end{frame}

\begin{frame}{Definition of probability - motivation}
\begin{itemize}
 \item probability $p$ assign a number from $[0,1]$ to given event,\\
       it is \df{measure}\cz{míra} of a subset $\omega \subset \Omega$
 \item for infinite $\Omega$ we can not measure all subsets of $\Omega$ !!
 \item always can measure: $\Omega$, $\emptyset$
 \item can measure unions 
 \item given set of \df{measurable}\cz{měřitelných} events\\
       forms \df{system of random events}
\end{itemize}
 
\end{frame}


\begin{frame}{System of random events}

\begin{definition}
\df{System of random events} ($\sigma$-algebra) $\mathcal{S}(\Omega)$ is a system of subsets of $\Omega$ satisfying
\begin{enumerate}
\item  $\Omega \in \mathcal{S}$

\item
countable union of events from $\mathcal S$ belongs also to $\mathcal S$, i.e.
\[
\text{if } A_i\in \mathcal{S} \text{ then } \bigcup_{i=1}^{\infty} A_i \in \mathcal{S}
\]
%
\item complementary event is also from $\mathcal S$
\[
\text{if } A \in \mathcal{S} \text{ then } A^c \in \mathcal{S}
\]
\end{enumerate}
\end{definition}

Other properties can be proved, e.g.
\begin{itemize}
 \item \df{impossible event} belongs to $\mathcal S$; $\emptyset \in A$ since $\Omega^c \in \mathcal{S}$
 \item intersection of events belongs to $\mathcal S$; $A, B \in \mathcal{S} \impl A\cap B \in \mathcal{S}$ since $A\cap B = (A^c \cup B^c)^c$
\end{itemize}
\end{frame}

\begin{frame}{Probability as a measure}
We introduce \df{probability} as a measure, i.e. set function
\[
   P:  A\in \mathcal{S} \longrightarrow [0,1]
\]
where function $P$ must satisfy:
\begin{align}
 \tag{P1}&
 A\in \mathcal{S} \impl  P(A)\ge 0\\
 \tag{P2}&
 A_i\in \mathcal{S} \text{ is system of pairwise disjoint sets, then}\\
    \notag        
     &\qquad P\Big(\bigcup_{i= 1}^{\infty} A_i\Big) = \sum_{i=1}^{\infty} P(A_i)\\
 \tag{P3}&
 P(\Omega) = 1
\end{align}
\end{frame}

\begin{frame}{Probability space}
We can prove further properties of $P$:
\begin{itemize}
 \item $P(\emptyset) = 0$\\
  \blue{\it proof:} $P(\Omega) = P(\emptyset \cup \Omega) = P(\emptyset) + P(\Omega)$
  
 \xskip 
 \item if $A \subset B$ then $P(A)\le P(B)$\\
  \blue{\it proof:} $P(B) = P(A \cup B\setminus A)  = P(A) + P(B\setminus A) \ge P(A)$
 
 \xskip
 \item $P(A^c) = 1-P(A)$\\
  \blue{\it proof:} $P(A^c) +P(A) = P(A^c \cup A) = P(\Omega) = 1$
\end{itemize}

\xskip
if $P(A)=1$, we say that $A$ is \df{almost sure event}\cz{skoro jistý jev}

The triple $(\Omega, \mathcal{S}, P)$ forms the \df{probability space}\cz{pravděpodobnostní prostor}.
\end{frame}


\begin{frame}{Inclusion exclusion principle}
\blue{simple:}
\[
    P(A\cup B) = P(A) + P(B) - P(A\cap B)
\]
\blue{general:}
\[
    P(\cup A_i) = \sum P(A_{i})  - \sum P(A_{i}\cap A_{j}) + \sum P(A_{i}\cap A_{j} \cap A_{k}) -\ \dots
\]

\xskip
 \blue{1)} probability that number from $S=\{1,\dots,100\}$ is divisible by $6$ or $7$?\\
  $16$ divisible by $6$; $14$ divisible by $7$; $2$ divisible by both ($6\times 7 = 42$)
  \[
     P(A_6 \cup A_7) = P(A_6) + P(A_7) - P(A_6 \cap A_7) = \frac{16}{100} + \frac{14}{100} - \frac{2}{100} = \frac{28}{100} = 0.28
  \]
\blue{2)} What is probability that number from $S$ is divisible by $3$ or  $4$ or $5$?
\begin{align*}
  P(A_3 \cup A_4 \cup A_5) =& P(A_3) + P(A_4) + P(A_5)\\
              &- P(A_{34}) - P(A_{45}) - P(A_{35})\\
               &+ P(A_{345})=\frac{6}{10} 
\end{align*}
\end{frame}


\begin{frame}{Finite probability space}

\begin{itemize}
\item model for qualitative and discrete empirical variables\\
\item \blue{examples of $\Omega$:} types of defects of a product, political parties, dice
\item $\Omega={1, \dots ,n}$
\item system $\mathcal{S} = 2^\Omega$, contains all subsets
\item probability of elementary events:\\
      $P(i) = p_i$, for $i\in \Omega$,\\
      where $p_i$ are non-negative numbers and their sum is $1$, \\
\item for general event $M$:\\      
      $M = \{1,4,5\}$ is  $P(M) = p_1 + p_4 + p_5$    

\item Note: $\binom{n}{j}$ is number of events with $j$ elementary events\\
\end{itemize}
\end{frame}       

\begin{frame}{Countable \cz{spočetný} probability spaces}
\begin{itemize}
\item \blue{examples:} Probability that number of nuclear decays (in big sample) during one second will be $k$.
\item $\Omega = 1,2, \dots$\\
\item again $\mathcal{S} = 2^\Omega$,\\
\item again probabilities $p_i$ of elementary events, but now forming a sequence that should sum to $1$:\\
       \[
       \sum_{i=1}^\infty p_i = 1
       \]
\end{itemize}
\end{frame}


\begin{frame}{Uncountable \cz{nespočetný} probability spaces}
\begin{itemize}
\item \blue{Example:} all imaginable people, all possible heights or weights of people
\item $\Omega = \Real$, $\Real^N$, $C(K)$
\item $\mathcal{S}$ can be  \df{Borel system} of sets on real axis $\mathcal B(\Real)$, that is minimal system 
       containing all intervals
\item general probability measure;\\
on $\Real$ we can use  \df{distribution function}\cz{distribuční funkce} $F$\\
For interval $[a,b)$ we define
\[
  P([a,b)) = F(a)- F(b)
\]
probability of unions, intersections, and complements (every Borel set $A$ can be written through them)
is defined by (P2)

That probability can be written as integral:
\[
  P(A) = \int_{A} F'(t) \dt
\]
\end{itemize}
\end{frame}

\begin{frame}{Probability space - abstraction}
\begin{itemize}
\item Consider survey asking people for: sex $S$, weight $W$, height $H$
\item has to consider probability space $\Omega = \Omega_{S} \times \Omega_{W} \times \Omega_{H}$
\item choice of the space depends on the variables we are observing
\item unpractical \dots "abstract events", variables as mappings from abstract space to the space of the variable
\end{itemize}
\end{frame}



\begin{frame}{Conditional probability}
\[
P(A|B) = \frac{P(A\cap B)}{P(B)}
\]
\begin{itemize}
\item theorem about \blue{multiplication of probabilities}
\[
P(A\cap B) = P(A|B) P(B)
\]
\[
P(A_{1\dots n}) = P(A_1) P(A_2|A_1) P(A_3|A_{12}) \dots P(A_n|A_{1\dots n-1})
\]

\item restricted space $( \Omega, \mathcal S, P(\cdot| B))$ 
\item theorem about \blue{total probability}
\[
  P(A) = \sum_i P(D_i) P(A|D_i)
\]
for any disjoint sets $D_i$ that cover $\Omega$, $\Omega = \cup_i D_i$
\end{itemize}
\end{frame}



\begin{frame}{Bayes' theorem}
Simplest form:
\[
 P(B| A) = \frac{P(B) P(A|B)}{P(A)}
\]

\xskip
General form for disjoint sets $D_i$ covering $\Omega$:
\[
 P(D_i| A) = \frac{P(D_i) P(A|D_i)}{P(A)} = \frac{P(D_i) P(A|D_i)}{\sum_k P(A|D_k)P(D_k)}
\]

\end{frame}


\begin{frame}{Example}
Let us consider an disease with prevalence $1\%$.\\
Screening gives the correct result in $95\%$ for a person with disease,\\
and in $70\%$ for a person without the disease.
Determine the probability of correct diagnosis for the case of positive and for the negative 
result of screening.

\pause
\xskip
event $A=$ the person has disease,\\
event $B=$ result is positive (should indicate $A$)\\
from setting: $P(A) = 0.01$, $P(B| A) = 0.95$, $P(B^c|A^c)=0.7$

\xskip
Probability of correct result for positive screening:
   \[
     P(A| B) = \frac{P(B| A) P(A)}{P(B| A)P(A) + P(B| A^c)P(A^c)}=
 \frac{P(B| A) P(A)}{P(B)} = \frac{0.0095}{0.3065}=0.031
   \]

Probability of correct result for negative screening:
   \[
     P(A^c| B^c) =
 \frac{P(B^c| A^c) P(A^c)}{P(B^c)} = \frac{0.693}{0.6935}=0.9993
   \]
\end{frame}

\begin{frame}{Independent events}
\begin{definition}
Events $A$ and $B$ are \df{independent} if $P(A\cap B) = P(A)P(B)$.
\end{definition}

\xskip
motivated by:\\
$P(A) = P(A|B)=P(A | B^c)$ or $P(B) = P(B|A)=P(B | A^c)$

\xskip
Events $A_1$, $A_2$, \dots, $A_n$ are \df{mutually independent} if for every subset of 
indexes $M \subset \{1,\dots,n\}$, holds
\[
P\big(\bigcap_{i\in M} A_i\big) = \prod_{i\in M} P(A_i)
\]

\end{frame}





  
\end{document}



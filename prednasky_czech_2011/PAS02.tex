
% $Header: /cvsroot/latex-beamer/latex-beamer/solutions/generic-talks/generic-ornate-15min-45min.en.tex,v 1.5 2007/01/28 20:48:23 tantau Exp $

\documentclass[smaller]{beamer}
\mode<presentation>
{
  \usetheme{Singapore}
  \usefonttheme[onlymath]{serif}
  % or ...
 %  \setbeamercovered{transparent}
  % or whatever (possibly just delete it)
}


\usepackage[czech]{babel}
% or whatever
\usepackage[utf8]{inputenc}
% or whatever
%\usepackage{times}
%\usepackage[T1]{fontenc}
% Or whatever. Note that the encoding and the font should match. If T1
% does not look nice, try deleting the line with the fontenc.


\title{PAS02 - základní pojmy teorie pravděpodobnosti}

\author{Jan B\v rezina}
\institute % (optional, but mostly needed)
{
  %\inst{2}%
  Technical University of Liberec
}


% If you wish to uncover everything in a step-wise fashion, uncomment
% the following command: 

%\beamerdefaultoverlayspecification{<+->}

% ***************************************** SYMBOLS
\def\div{{\rm div}}
\def\Lapl{\Delta}
\def\grad{\nabla}
\def\supp{{\rm supp}}
\def\dist{{\rm dist}}
%\def\chset{\mathbbm{1}}
\def\chset{1}

\def\Tr{{\rm Tr}}
\def\sgn{{\rm sgn}}
\def\to{\rightarrow}
\def\weakto{\rightharpoonup}
\def\imbed{\hookrightarrow}
\def\cimbed{\subset\subset}
\def\range{{\mathcal R}}
\def\leprox{\lesssim}
\def\argdot{{\hspace{0.18em}\cdot\hspace{0.18em}}}
\def\Distr{{\mathcal D}}
\def\calK{{\mathcal K}}
\def\FromTo{|\rightarrow}
\def\convol{\star}
\def\impl{\Rightarrow}
\DeclareMathOperator*{\esslim}{esslim}
\DeclareMathOperator*{\esssup}{ess\,sup}
\DeclareMathOperator{\ess}{ess}
\DeclareMathOperator{\osc}{osc}
\DeclareMathOperator{\curl}{curl}

%\def\Ess{{\rm ess}}
%\def\Exp{{\rm exp}}
%\def\Implies{\Longrightarrow}
%\def\Equiv{\Longleftrightarrow}
% ****************************************** GENERAL MATH NOTATION
\def\Real{{\rm\bf R}}
\def\Rd{{{\rm\bf R}^{\rm 3}}}
\def\RN{{{\rm\bf R}^N}}
\def\D{{\mathbb D}}
\def\Nnum{{\mathbb N}}
\def\Measures{{\mathcal M}}
\def\d{\,{\rm d}}               % differential
\def\sdodt{\genfrac{}{}{}{1}{\rm d}{{\rm d}t}}
\def\dodt{\genfrac{}{}{}{}{\rm d}{{\rm d}t}}

\def\vc#1{\mathbf{\boldsymbol{#1}}}     % vector
\def\tn#1{{\mathbb{#1}}}    % tensor
\def\abs#1{\lvert#1\rvert}
\def\Abs#1{\bigl\lvert#1\bigr\rvert}
\def\bigabs#1{\bigl\lvert#1\bigr\rvert}
\def\Bigabs#1{\Big\lvert#1\Big\rvert}
\def\ABS#1{\left\lvert#1\right\rvert}
\def\norm#1{\bigl\Vert#1\bigr\Vert} %norm
\def\close#1{\overline{#1}}
\def\inter#1{#1^\circ}
\def\ol#1{\overline{#1}}
\def\ul#1{\underline{#1}}
\def\eqdef{\mathrel{\mathop:}=}     % defining equivalence
\def\where{\,|\,}                    % "where" separator in set's defs
\def\timeD#1{\dot{\overline{{#1}}}}

% ******************************************* USEFULL MACROS
\def\RomanEnum{\renewcommand{\labelenumi}{\rm (\roman{enumi})}}   % enumerate by roman numbers
\def\rf#1{(\ref{#1})}                                             % ref. shortcut
\def\prtl{\partial}                                        % partial deriv.
\def\Names#1{{\scshape #1}}
\def\rem#1{{\parskip=0cm\par!! {\sl\small #1} !!}}

\def\Xint#1{\mathchoice
{\XXint\displaystyle\textstyle{#1}}%
{\XXint\textstyle\scriptstyle{#1}}%
{\XXint\scriptstyle\scriptscriptstyle{#1}}%
{\XXint\scriptscriptstyle\scriptscriptstyle{#1}}%
\!\int}
\def\XXint#1#2#3{{\setbox0=\hbox{$#1{#2#3}{\int}$}
\vcenter{\hbox{$#2#3$}}\kern-.5\wd0}}
\def\ddashint{\Xint=}
\def\dashint{\Xint-}

% ******************************************* DOCUMENT NOTATIONS
% document specific
\def\rh{\varrho}
\def\vl{{\vc{u}}}
\def\th{\vartheta}
\def\vx{\vc{x}}
\def\vX{\vc{X}}
\def\vr{\vc{r}}
\def\veta{\vc{\eta}}
\def\dx{\,\d\vx}
\def\dt{\,\d t}
\def\bulk{\zeta}
\def\cS{\close{S}}
\def\eps{\varepsilon}
\def\phi{\varphi}
\def\Bog{{\mathcal B}}
\def\Riesz{{\mathcal R}}
\def\distr{\mathcal D}
\def\Item{$\bullet$}

\def\MEtst{\mathcal T}
%***************************************************************************
\setbeamercolor{my blue}{fg=blue}
\def\blue#1{{\usebeamercolor[fg]{my blue} #1}}
\setbeamercolor{my green}{fg=green}
\def\green#1{{\usebeamercolor[fg]{my green} #1}}
\setbeamercolor{my red}{fg=red}
\def\red#1{{\usebeamercolor[fg]{my red} #1}}
\def\xskip{{\vspace{2ex}}}

\begin{document}

\begin{frame}
  \titlepage
\end{frame}

% zvýraznění definovaného pojmu
\def\df{\usebeamercolor[fg]{my red}\it}

\begin{frame}{Co je to pravděpodobnost?}
Jaká je pravděpodobnost, že uspěju u zkoušky? \pause \\
(empiriská četnost)\\
Jaká je pravděpodobnost, že mi padne šestka (na kostce)? \pause \\
(model - předpokládá spravedlivou kostku)\\
Jaká je pravděpodobnost, že mi padne sudé číslo (na kostce)? \pause \\
(klasická pravděpodobnost)\\
Jaká je pravděpodobnost, že se v šipkách trefím do středu?\pause \\
(empirická četnost nebo geometrická pravděpodobnost)\\
Jaká je pravděpodobnost, že vás zasáhne troska padající družice? \pause \\
(pravděpodobnost jako množinová funkce)\\
\end{frame}

\begin{frame}{Náhodný pokus, systém jevů}
\blue{Náhodný pokus} - vygenerování prvku množiny $\Omega$ všech možných výsledků, též \blue{prostor pozorování} nebo \blue{prostor elementárních jevů}

Př.  vytahování ponožek z koše $\Omega=\{$bílá$, $šedivá$, $černá$\}$, \\
     pro hod kostkou je $\Omega = \{1,2,3,4,5,6\}$,\\
     pro měření teploty $\Omega = (0,\infty)$,\\
     aktuální směr a rychlost větru $\Omega= \Real^3$,\\
     průběh vektoru rychlosti během jednoho dne $\Omega = C(0,24)$,\\
     aktualní pole vlhkosti na území Liberce $\Omega = C( K )$, $K \subset \Real^2$\\
\xskip
\blue{jev} - podmnožina $\Omega$, též lze chápat jako nějaké tvrzení o výsledku náhodného pokusu

Př. nevytáhnu šedou ponožku, padne liché číslo, \\
    teplota bude z intervalu $(273,373)$ kelvinů, \\
    vítr bude v severo--jižním směru

\blue{elementární jevy} - prvky množiny $\Omega$
\end{frame}

\begin{frame}{Vztahy a operace s jevy = množinami}
Vztahy mezi jevy:
\begin{align*}
 A\subset B&\quad\text{pokud nastane jev  $A$ pak nastane i jev $B$}\\
 A = B  &\quad\text{$A$ právě tehdy když $B$}\\
 A\cap B =\emptyset&\quad\text{disjunktní jevy, jeden vylučuje druhý}
\end{align*} 

Operace s jevy:
\begin{align*}
 A\cup B  &\quad\text{nastane $A$ nebo $B$}\\
 A\cap B  &\quad\text{nastane $A$ a zároveň $B$}\\
 \overline{A}, A^c &\quad\text{nenastane $A$},\text{ jev opačný}\\
 A\setminus B &\quad\text{nastane $A$ a nenastane $B$}\\
 A\bigtriangleup B &\quad\text{nastane $A$ nebo $B$, ale ne obojí}
\end{align*}

Též spočetná sjednocení a průniky:
\[
  \bigcup_{i\in N} A_i,\quad \bigcap_{i\in N} A_i
\]
\end{frame}


\begin{frame}{Systém náhodných jevů - přesněji}

\begin{definition}
Systém náhodných jevů $\mathcal{S}(\Omega)$ je systém podmnožin množiny $\Omega$ splňující
\begin{enumerate}
\item 
\blue{Jev jistý} patří do 
$\mathcal{S}$, t.j. $\Omega \in \mathcal{S}$

\item
Sjednocení jevů z $\mathcal S$ patří též do $\mathcal S$.
\[
\text{pokud } A_i\in \mathcal{S} \text{ pak také } \bigcup_{i=1}^{\infty} A_i \in \mathcal{S}
\]
%
\item Jev opačný je opět z $\mathcal S$
\[
\text{pokud } A \in \mathcal{S} \text{ pak také } A^c \in \mathcal{S}
\]
\end{enumerate}
\end{definition}

Další vlastnosti se dokazují, např. :
\begin{itemize}
 \item nemožný jev patří do $\mathcal S$, $\emptyset \in A$ jelikož $\Omega^c \in \mathcal{S}$
 \item průnik jevů patří do $\mathcal S$, $A, B \in \mathcal{S} \impl A\cap B \in \mathcal{S}$ jelikož $A\cap B = (A^c \cup B^c)^c$
\end{itemize}
\end{frame}

\begin{frame}{pravděpodobnost jako míra}
\blue{Pravděpodobnost} zavedeme jako míru, tj. množinovou funkci
\[
   P:  A\in \mathcal{S} \longrightarrow [0,1]
\]
přitom funkce $p$ musí splňovat:
\begin{align}
 \tag{P1}&
 A\in \mathcal{S} \impl  P(A)\ge 0\\
 \tag{P2}&
 A_i\in \mathcal{S} \text{ je spočetný systém podou disjunktních množin, pak}\\
    \notag        
     &\qquad P\Big(\bigcup_{i= 1}^{\infty} A_i\Big) = \sum_{i=1}^{\infty} P(A_i)\\
 \tag{P3}&
 P(\Omega) = 1
\end{align}
\end{frame}

\begin{frame}{Pravděpodobnostní prostor}
Další dokazované vlastnosti:
\begin{itemize}
 \item $P(\emptyset) = 0$
 \item pokud $A \subset B$ pak $P(A)\le P(B)$
 \item $P(A^c) = 1-P(A)$
 \item $P(A\cup B) = P(A) + P(B) - P(A\cap B)$
\end{itemize}

\xskip
pokud $P(A)=1$, říkáme, že jev $A$ nastane \blue{skoro jistě}

Trojici $(\Omega, \mathcal{S}, P)$, říkáme \blue{pravděpodobnostní prostor}.
\end{frame}

\begin{frame}{Typy pravděpodobnostních prostorů}
\begin{enumerate}
 \item $\Omega$ je konečná množina, př. $\Omega={1, \dots ,n}$,\\
       $\mathcal{S} = 2^\Omega$, tj. všechny podmnožiny,\\
       $P(i) = p_i$, kde $p_i$ jsou nezáporná čísla a jejich součet je $1$, \\
       např. pro jev $M = \{1,4,5\}$ je
       $P(M) = p_1 + p_4 + p_5$\\
       $\binom{n}{j}$ je počet jevů majících právě $j$ elementárních jevů\\
       model pro kvalitativní a konečné diskrétní empirické proměnné\\
       Př: \\
       hody kostkami (tam dokonce $p_i = 1/n$), druhy vad na výrobcích
       

 \item $\Omega$ je spočetná, př. $\Omega = N$\\
       opět $\mathcal{S} = 2^\Omega$,\\
       opět pravděp. elem. jevů $p_i$, ale tvoří nekon. posloupnost a tedy musí platit:\\
       \[
       \sum_{i=1}^\infty p_i = 1
       \]
       Př:\\
       Počet jader, které se rozpadnou za jednotku času. (ve skutečnosti jde take o konečnou proměnnou)
\end{enumerate}
\end{frame}


\begin{frame}{$\Omega$ nespočetná}
př. $\Omega = \Real$, $\Real^N$, $C(K)$, nebo nejaká podmnožina těchto množin

\xskip
       $\mathcal{S}$ je např. \blue{Borelovský systém} množin na přímce $\mathcal B(\Real)$, je to minimální systém
       jevů, který obsahuje všechny intervaly, tedy jejich spočetná sjednocení a průniky.
\xskip
zavedení pravděpodobnosti: \\
$F$ lib. neklesající funkce, spojitá zleva, s limitami $0$ v $-\infty$ a $1$ v $\infty$\\
tzv. \blue{distribuční funkce}\\
Pro interval $[a,b)$ definujeme
\[
  P([a,b)) = F(a)- F(b)
\]
pst. spočetných sjednocení disjunktních intervalů definujeme podle (P2)

Pst. lze také vyjádřit jako integrál ze zobecněné derivace distr. fce.
\[
  P(A) = \int_{A} F'(t) \dt
\]
\end{frame}

\begin{frame}{Podmíněná pravděpodobnost}
\[
P(A|B) = \frac{P(A\cap B)}{P(B)}
\]
\begin{itemize}
\item příjklady (Volf)
\item věta o \blue{násobení pravděpodobnosti}
\[
P(A\cap B) = P(A|B) P(B)
\]
\item věta o úplné pravděpodobnosti
\end{itemize}
\end{frame}



\begin{frame}{Bayesův vzorec}
Př. nemoc A $\dots 1\%$ \\
    spolehlivost vyšetření je $95\%$ pokud osoba má $A$, a $70\%$ pokudn osoba nemá $A$.
    Určete pst. spravné diag. pro případ pozitivního i negativního výsledku testu.\\

    B - výsledek testu pozitivní.\\

   ze zadání: $P(A) = 0.01$, $P(B| A) = 0.95$, $P(B^c|A^c)=0.75$\\
   \[
     P(A| B) = \frac{P(B| A) P(A)}{P(B| A)P(A) + P(B| A^c)P(A^c)}=
 \frac{P(B| A) P(A)}{P(B)} = \frac{0.0095}{0.3065}=0.031
   \]

   \[
     P(A^c| B^c) =
 \frac{P(B^c| A^c) P(A^c)}{P(B^c)} = \frac{0.693}{0.6935}=0.9993
   \]
\end{frame}

\begin{frame}{Závislost jevů}
\begin{definition}
Jevy $A_1$, $A_2$, \dots, $A_n$ jsou \blue{vzájemě nezávislé} pokud pro každou 
podmnožinu indexů $M \subset \{1,\dots,n\}$, platí
\[
P\big(\bigcap_{i\in M} A_i\big) = \prod_{i\in M} P(A_i)
\]
\end{definition}
\end{frame}





  
\end{document}



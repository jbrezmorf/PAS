
% $Header: /cvsroot/latex-beamer/latex-beamer/solutions/generic-talks/generic-ornate-15min-45min.en.tex,v 1.5 2007/01/28 20:48:23 tantau Exp $

\documentclass[smaller]{beamer}
\mode<presentation>
{
  \usetheme{Singapore}
  \usefonttheme[onlymath]{serif}
  % or ...
 %  \setbeamercovered{transparent}
  % or whatever (possibly just delete it)
}

\usepackage{textcomp}
\usepackage[czech]{babel}
% or whatever
\usepackage[utf8]{inputenc}
% or whatever
%\usepackage{times}
%\usepackage[T1]{fontenc}
% Or whatever. Note that the encoding and the font should match. If T1
% does not look nice, try deleting the line with the fontenc.


\title{PAS07 -- Testovaní hypotéz}

\author{Jan B\v rezina}
\institute % (optional, but mostly needed)
{
  %\inst{2}%
  Technical University of Liberec
}


% If you wish to uncover everything in a step-wise fashion, uncomment
% the following command: 

%\beamerdefaultoverlayspecification{<+->}

% ***************************************** SYMBOLS
\def\div{{\rm div}}
\def\Lapl{\Delta}
\def\grad{\nabla}
\def\supp{{\rm supp}}
\def\dist{{\rm dist}}
%\def\chset{\mathbbm{1}}
\def\chset{1}

\def\Tr{{\rm Tr}}
\def\sgn{{\rm sgn}}
\def\to{\rightarrow}
\def\weakto{\rightharpoonup}
\def\imbed{\hookrightarrow}
\def\cimbed{\subset\subset}
\def\range{{\mathcal R}}
\def\leprox{\lesssim}
\def\argdot{{\hspace{0.18em}\cdot\hspace{0.18em}}}
\def\Distr{{\mathcal D}}
\def\calK{{\mathcal K}}
\def\FromTo{|\rightarrow}
\def\convol{\star}
\def\impl{\Rightarrow}
\DeclareMathOperator*{\esslim}{esslim}
\DeclareMathOperator*{\esssup}{ess\,sup}
\DeclareMathOperator{\ess}{ess}
\DeclareMathOperator{\osc}{osc}
\DeclareMathOperator{\curl}{curl}

%\def\Ess{{\rm ess}}
%\def\Exp{{\rm exp}}
%\def\Implies{\Longrightarrow}
%\def\Equiv{\Longleftrightarrow}
% ****************************************** GENERAL MATH NOTATION
\def\Real{{\rm\bf R}}
\def\Rd{{{\rm\bf R}^{\rm 3}}}
\def\RN{{{\rm\bf R}^N}}
\def\D{{\mathbb D}}
\def\Nnum{{\mathbb N}}
\def\Measures{{\mathcal M}}
\def\d{\,{\rm d}}               % differential
\def\sdodt{\genfrac{}{}{}{1}{\rm d}{{\rm d}t}}
\def\dodt{\genfrac{}{}{}{}{\rm d}{{\rm d}t}}

\def\vc#1{\mathbf{\boldsymbol{#1}}}     % vector
\def\tn#1{{\mathbb{#1}}}    % tensor
\def\abs#1{\lvert#1\rvert}
\def\Abs#1{\bigl\lvert#1\bigr\rvert}
\def\bigabs#1{\bigl\lvert#1\bigr\rvert}
\def\Bigabs#1{\Big\lvert#1\Big\rvert}
\def\ABS#1{\left\lvert#1\right\rvert}
\def\norm#1{\bigl\Vert#1\bigr\Vert} %norm
\def\close#1{\overline{#1}}
\def\inter#1{#1^\circ}
\def\ol#1{\overline{#1}}
\def\ul#1{\underline{#1}}
\def\eqdef{\mathrel{\mathop:}=}     % defining equivalence
\def\where{\,|\,}                    % "where" separator in set's defs
\def\timeD#1{\dot{\overline{{#1}}}}

% ******************************************* USEFULL MACROS
\def\RomanEnum{\renewcommand{\labelenumi}{\rm (\roman{enumi})}}   % enumerate by roman numbers
\def\rf#1{(\ref{#1})}                                             % ref. shortcut
\def\prtl{\partial}                                        % partial deriv.
\def\Names#1{{\scshape #1}}
\def\rem#1{{\parskip=0cm\par!! {\sl\small #1} !!}}

\def\Xint#1{\mathchoice
{\XXint\displaystyle\textstyle{#1}}%
{\XXint\textstyle\scriptstyle{#1}}%
{\XXint\scriptstyle\scriptscriptstyle{#1}}%
{\XXint\scriptscriptstyle\scriptscriptstyle{#1}}%
\!\int}
\def\XXint#1#2#3{{\setbox0=\hbox{$#1{#2#3}{\int}$}
\vcenter{\hbox{$#2#3$}}\kern-.5\wd0}}
\def\ddashint{\Xint=}
\def\dashint{\Xint-}

% ******************************************* DOCUMENT NOTATIONS
% document specific
\def\rh{\varrho}
\def\vl{{\vc{u}}}
\def\th{\vartheta}
\def\vx{\vc{x}}
\def\vX{\vc{X}}
\def\vr{\vc{r}}
\def\veta{\vc{\eta}}
\def\dx{\,\d\vx}
\def\dt{\,\d t}
\def\bulk{\zeta}
\def\cS{\close{S}}
\def\eps{\varepsilon}
\def\phi{\varphi}
\def\Bog{{\mathcal B}}
\def\Riesz{{\mathcal R}}
\def\distr{\mathcal D}
\def\Item{$\bullet$}

\def\MEtst{\mathcal T}
%***************************************************************************
\setbeamercolor{my blue}{fg=blue}
\def\blue#1{{\usebeamercolor[fg]{my blue} #1}}
\setbeamercolor{my green}{fg=green}
\def\green#1{{\usebeamercolor[fg]{my green} #1}}
\setbeamercolor{my red}{fg=red}
\def\red#1{{\usebeamercolor[fg]{my red} #1}}
\def\xskip{{\vspace{2ex}}}
\def\E{\vc{\mathsf{E}}}

\begin{document}

\begin{frame}
  \titlepage
\end{frame}

% zvýraznění definovaného pojmu
\def\df{\usebeamercolor[fg]{my red}\it}

\begin{frame}{Motivace - problematický intervalový odhad pro relativní četnost}
Máme data $\{X_i\}$ z $Alt(\pi)$. Výběrová četnost je $p$.
Podle Moivreovy-Laplaceovy věty:
\[
U = \frac{p - \pi}{\sqrt{\pi(1-\pi}}\sqrt{n} \quad \sim N(0,1)
\]
Tedy 
\[
 P\big(u(\alpha/2) \le U \le u(1-\alpha/2)\big) = 1-\alpha
\]
po upravě:
\[
 P\big(p-\Delta < \pi < p+\Delta) = 1-\alpha
\]
\[
 \Delta = \sqrt{\frac{\pi(1-\pi)}{n}}u(1-\alpha/2) \pause \approx \sqrt{\frac{p(1-p)}{n}}u(1-\alpha/2)
\]
\end{frame}

\begin{frame}{Test hypotézy}
Př. Ve volbách získala ČSSD $30\%$ průzkum po půl roce, 50 repondentů, $25\%$. Lze usuzovat na pokles preferencí s pravděpodobností $80\%$?

nulová hypotéza (není pokles): $H_0: \pi = \pi_0 = 0.3$ \\
alternativní hypotéza (pokles): $H_A: \pi < \pi_0$

\begin{enumerate}
\item použitím intervalového odhadu (jednostranný), $\Delta = 0.049$, $p_0 > 0.25 +\Delta=0.29$, zamítneme $H_0$ ve prospěch $H_A$\pause NEPŘESNÉ
\pause
\item přímý test za předpokladu nulové hypotézy, známe $\pi = \pi_0$
\[
 U=\frac{\pi_0 - p}{\sqrt{\pi_0(1-\pi_0}}\sqrt{n}=\frac{0.3 - 0.25}{\sqrt{0.3(1-0.3}}\sqrt{50} = 0.77
\]
$u(0.8) \approx 0.84 > U$, nezamítáme $H_0$ 

\end{enumerate}
\end{frame}

\begin{frame}{Obecné schéma testování hypotéz (klasický test)}
Statistiku sestrojíme tak, aby s rostoucí hodnotou skutečného parametru rosla hodnota statistiky !!
Pak si odpovídají nerovnosti alternativní hypotézy a podmínky pro kritický obor.
\end{frame}

\begin{frame}{Chyby I. a II. druhu}
\end{frame}

\begin{frame}{Čistý test významnosti (p-value)}

{\df p-value}: nejnižší hladina $\alpha$ na které zamítnu $H_0$

\xskip
\blue{$H_A: \pi < \pi_0$} $\quad H_0$ zamítáme pro:
\[ U_{obs} < u(1-\alpha) = F^{-1}_U (\alpha) \quad \text{resp.}\]
\[ p=F_U(U_{obs}) <\alpha \]

\xskip
\blue{$H_A: \pi > \pi_0$} $\quad H_0$ zamítáme pro:
\[ U_{obs} > u(\alpha) = F^{-1}_U (1-\alpha) \quad \text{resp.}\]
\[ p=1-F_U(U_{obs}) <\alpha\]

\xskip
\blue{$H_A: \pi \ne \pi_0$} $\quad H_0$ zamítáme pro:
\[ U_{obs} < F^{-1}_U (\alpha/2) \vee F^{-1}_U (1-\alpha/2)< U_{obs} \quad \text{resp.}\]
\[ p= 2\min\{1-F_U(U_{obs}) , F_U(U_{obs}) < \alpha\}\]   
\end{frame}

\begin{frame}{Test o střední hodnotě}
\end{frame}

\begin{frame}{Test o rozptylu}
\end{frame}

\begin{frame}{Test o relativní četnosti}
\end{frame}

\begin{frame}{Znaménkový test}
$\{X_1,\dots,X_n\}$ výběr ze spojitého rozdělení s mediánem $\tilde{x}$

$H_0: \tilde{x} = x_0$, a klasické tři alternativy

\begin{enumerate}
 \item z výběru vynecháme případy $X_i = x_0$ (zmenší se $n$)
 \item spočteme $Y$ --- počet případů $X_i -x_0 > 0$
 \item $Y$ má rozdělení $Bi(n, 1/2)$
 \item \dots čistý test významnosti \dots
 \item Pomocí $Y$ můžeme konstruovat intervalový odhad pro medián.
\end{enumerate}
\end{frame}
\begin{frame}{Znaménkový test (pokračování)}
Výpočet $p$-value
\begin{enumerate}
 \item pro malá $n$ spočítáme $p$-value přímo z definice (např. pro oboustrannou alternativu):
  \[
     p=2 \min\{1-F_Y(Y_{obs}), F_Y(Y_{obs})\}, \qquad F_Y(Y) = \sum_{k=0}^{Y} \binom{n}{k} (1/2)^n
  \]
 \item pro větší $n$ můžeme použít ekvivalenci Binomického s Fisherovým rozdělením:
  \[
     F_{Bi(n,p)}(s) = F_{2(n-s),2(s+1)} \Big(\frac{(s+1)(1-p)}{p(n-s)}\Big)
  \]
 \item pro velká $n$ (Moivrova věta $np(1-p) >9$) použijeme aproximaci normálním rozdělením:
 \[
    U = \frac{Y - np}{\sqrt{np(1-p)}} = \frac{2Y -n}{\sqrt{n}} \sim N(0,1)
 \]
\end{enumerate}
TODO: stabilizující transformace
\end{frame}

\begin{frame}{Wilcoxonův test}
$\{X_1,\dots,X_n\}$ výběr ze {\it symmetrického} ($f(\tilde{x} + x) = f(\tilde{x} -x)$ spojitého rozdělení s mediánem $\tilde{x}$.

$H_0: \tilde{x} = x_0$, a klasické tři alternativy

\begin{enumerate}
 \item odstraníme případy $X_i = x_0$ a označíme $Y_i = X_i - x_0$
 \item setřídíme $\{Y_i\}$ podle $\abs{Y_i}$
 \item označme $R_i$ poředí hodnoty $Y_i$ v setříděné posloupnosti
 \item \[
          S^+ = \sum_{Y_i >0} R_i\quad  S^- = \sum_{Y_i <0} R_i\quad S^+ + S^- = n(n+1)/2
       \]
 \item $Y = \min\{S^+, S^-\}$
 
\end{enumerate}

\end{frame}

\begin{frame}{Wilcoxonův test (pokračování)}
 \begin{itemize}
  \item Pro malé hodnoty $Y$, zamítáme $H_0$ ve prospěch $H_A: \tilde{x} \ne x_0$ pokud $Y \le w_n(\alpha)$ (tabelované kritické hodnoty)
  \item Pro větší hodnty $n$ použijeme aproximaci:
  \[
    U = \frac{S^+ - ES^+}{\sqrt{var S^+}} = \frac{12 S}{n(n+1)(2n+1)} \sim N(0,1)
  \]
kde 
  \[
    ES^+ = \frac{1}{4}n(n+1),\quad var S^+ = \frac{1}{24}n(n+1)(2n+1)
  \]
  \[
    S = S^+ - S^- = \sum_{i} R_i \sgn Y_i
  \]
 \end{itemize}
\end{frame}


\begin{frame}{Příkad na testy mediánu}
10 osob mělo odhadnout dobu 1 minuta. $H_0: \tilde{x} = x_0$, $H_A: \tilde{x} \ne x_0$
\[
 X_i: 53, 48, 45, 55, 63, 51, 66, 56, 50, 58
\]
\[
 Y_i: -7, -12, -15, -5, 3, -9, 6, -4, -10, -2
\]

%\begin{enumerate}

%\end{enumerate}

\end{frame}


\begin{frame}{Test o distribuční funkci - Kolmogorov-Smirnov}
  Předpoklady: 
  \begin{itemize}
   \item spojité rozdělení
   \item citlivé více v okolí mediánu
   \item plně zadaná distribuční funkce
  \end{itemize}

\xskip
  $H_0$ : data $X_n$ mají distribuční funkci $F(x)$\\
  $H_1$ : data nemají distribuční funkci $F(x)$
\end{frame}

\begin{frame}{Provedení testu K-S}
  \begin{enumerate}
   \item Sestavení empirické distribuční funkce $F_n(x)$.
   \item Výpočet hodnoty:
   \[
      D_n = \sup_{x} \abs{F_n(x) - F(x)} = \max_{i} \big( \abs{ \frac{i-1}{n} - F(X_n)}, \abs{\frac{i}{n} - F(X_n)} \big)
   \]
   \item Pro malá $n$ porovnáme s přesnými kritickými hodnotami $D_n(\alpha)$, viz skripta Ostrava, Tabulka 5. $H_0$ se zamítá pokud $D_n \ge D_n(\alpha)$.
   \item Pro velká $n$ použijeme asymptotické kritické hodnoty podle vzorce:
         \[
            D_n^{\star}(\alpha) = \sqrt{\frac{1}{2n} \ln\frac{2}{\alpha}}
         \]
  \end{enumerate}

\end{frame}

\begin{frame}{příklad}
 Testujeme genrátor náhodných čísel na rozdělení $R(0,1)$.
\[
X_i: .93, .35, .66, .93, .14, .23, .08, .23, .21, .59
\]


\end{frame}


\begin{frame}{Testy o normalitě rozdělení}
 {\df Lilliefors test} - provádí se jako K-S, ale parametry normálního rozdělení se určí pomocí bodových odhadů $\overline X_n$ a $S_n$.
 Pro kritické hodnoty je třeba použít modifikované tabulky.
\end{frame}

\begin{frame}{Párové testy}
\end{frame}

\end{document}




% $Header: /cvsroot/latex-beamer/latex-beamer/solutions/generic-talks/generic-ornate-15min-45min.en.tex,v 1.5 2007/01/28 20:48:23 tantau Exp $

\documentclass[smaller]{beamer}
\mode<presentation>
{
  \usetheme{Singapore}
  \usefonttheme[onlymath]{serif}
  % or ...
 %  \setbeamercovered{transparent}
  % or whatever (possibly just delete it)
}


\usepackage[czech]{babel}
% or whatever
\usepackage[utf8]{inputenc}
% or whatever
%\usepackage{times}
%\usepackage[T1]{fontenc}
% Or whatever. Note that the encoding and the font should match. If T1
% does not look nice, try deleting the line with the fontenc.


\title{PAS06 -- Odhady parametrů}

\author{Jan B\v rezina}
\institute % (optional, but mostly needed)
{
  %\inst{2}%
  Technical University of Liberec
}


% If you wish to uncover everything in a step-wise fashion, uncomment
% the following command: 

%\beamerdefaultoverlayspecification{<+->}

% ***************************************** SYMBOLS
\def\div{{\rm div}}
\def\Lapl{\Delta}
\def\grad{\nabla}
\def\supp{{\rm supp}}
\def\dist{{\rm dist}}
%\def\chset{\mathbbm{1}}
\def\chset{1}

\def\Tr{{\rm Tr}}
\def\sgn{{\rm sgn}}
\def\to{\rightarrow}
\def\weakto{\rightharpoonup}
\def\imbed{\hookrightarrow}
\def\cimbed{\subset\subset}
\def\range{{\mathcal R}}
\def\leprox{\lesssim}
\def\argdot{{\hspace{0.18em}\cdot\hspace{0.18em}}}
\def\Distr{{\mathcal D}}
\def\calK{{\mathcal K}}
\def\FromTo{|\rightarrow}
\def\convol{\star}
\def\impl{\Rightarrow}
\DeclareMathOperator*{\esslim}{esslim}
\DeclareMathOperator*{\esssup}{ess\,sup}
\DeclareMathOperator{\ess}{ess}
\DeclareMathOperator{\osc}{osc}
\DeclareMathOperator{\curl}{curl}

%\def\Ess{{\rm ess}}
%\def\Exp{{\rm exp}}
%\def\Implies{\Longrightarrow}
%\def\Equiv{\Longleftrightarrow}
% ****************************************** GENERAL MATH NOTATION
\def\Real{{\rm\bf R}}
\def\Rd{{{\rm\bf R}^{\rm 3}}}
\def\RN{{{\rm\bf R}^N}}
\def\D{{\mathbb D}}
\def\Nnum{{\mathbb N}}
\def\Measures{{\mathcal M}}
\def\d{\,{\rm d}}               % differential
\def\sdodt{\genfrac{}{}{}{1}{\rm d}{{\rm d}t}}
\def\dodt{\genfrac{}{}{}{}{\rm d}{{\rm d}t}}

\def\vc#1{\mathbf{\boldsymbol{#1}}}     % vector
\def\tn#1{{\mathbb{#1}}}    % tensor
\def\abs#1{\lvert#1\rvert}
\def\Abs#1{\bigl\lvert#1\bigr\rvert}
\def\bigabs#1{\bigl\lvert#1\bigr\rvert}
\def\Bigabs#1{\Big\lvert#1\Big\rvert}
\def\ABS#1{\left\lvert#1\right\rvert}
\def\norm#1{\bigl\Vert#1\bigr\Vert} %norm
\def\close#1{\overline{#1}}
\def\inter#1{#1^\circ}
\def\ol#1{\overline{#1}}
\def\ul#1{\underline{#1}}
\def\eqdef{\mathrel{\mathop:}=}     % defining equivalence
\def\where{\,|\,}                    % "where" separator in set's defs
\def\timeD#1{\dot{\overline{{#1}}}}

% ******************************************* USEFULL MACROS
\def\RomanEnum{\renewcommand{\labelenumi}{\rm (\roman{enumi})}}   % enumerate by roman numbers
\def\rf#1{(\ref{#1})}                                             % ref. shortcut
\def\prtl{\partial}                                        % partial deriv.
\def\Names#1{{\scshape #1}}
\def\rem#1{{\parskip=0cm\par!! {\sl\small #1} !!}}

\def\Xint#1{\mathchoice
{\XXint\displaystyle\textstyle{#1}}%
{\XXint\textstyle\scriptstyle{#1}}%
{\XXint\scriptstyle\scriptscriptstyle{#1}}%
{\XXint\scriptscriptstyle\scriptscriptstyle{#1}}%
\!\int}
\def\XXint#1#2#3{{\setbox0=\hbox{$#1{#2#3}{\int}$}
\vcenter{\hbox{$#2#3$}}\kern-.5\wd0}}
\def\ddashint{\Xint=}
\def\dashint{\Xint-}

% ******************************************* DOCUMENT NOTATIONS
% document specific
\def\rh{\varrho}
\def\vl{{\vc{u}}}
\def\th{\vartheta}
\def\vx{\vc{x}}
\def\vX{\vc{X}}
\def\vr{\vc{r}}
\def\veta{\vc{\eta}}
\def\dx{\,\d\vx}
\def\dt{\,\d t}
\def\bulk{\zeta}
\def\cS{\close{S}}
\def\eps{\varepsilon}
\def\phi{\varphi}
\def\Bog{{\mathcal B}}
\def\Riesz{{\mathcal R}}
\def\distr{\mathcal D}
\def\Item{$\bullet$}

\def\MEtst{\mathcal T}
%***************************************************************************
\setbeamercolor{my blue}{fg=blue}
\def\blue#1{{\usebeamercolor[fg]{my blue} #1}}
\setbeamercolor{my green}{fg=green}
\def\green#1{{\usebeamercolor[fg]{my green} #1}}
\setbeamercolor{my red}{fg=red}
\def\red#1{{\usebeamercolor[fg]{my red} #1}}
\def\xskip{{\vspace{2ex}}}
\def\E{\vc{\mathsf{E}}}

\begin{document}

\begin{frame}
  \titlepage
\end{frame}

% zvýraznění definovaného pojmu
\def\df{\usebeamercolor[fg]{my red}\it}

\begin{frame}{Motivační otázky}
 \begin{itemize}
  \item Měříme data u nichž předpokládáme normální rozdělení, ale neznáme jeho parametry $\mu$ a $\sigma^2$.
    Můžeme tyto parametry nějak odhadnout z naměřených dat? Jak přesně?
  \item Provádíme kontrolu kvality. Kolik chyb z dodávky 100 výrobků můžeme zakazníkovi deklarovat, aby jich bylo více s pstí. max. 1\%?
        Tj. Binomické rozdělení, neznáme parametr $p$, ale zajímá nás $1\%$ kvantil.
  \item Porovnáváme dva léky na srážlivost krve. Jaký je rozdíl v jejich účinnosti, tj. rozdíl středních hodnot?
 \end{itemize}

\end{frame}


\begin{frame}{Bodový odhad}
Náhodný vektor $\vc X$ (nezávislých veličin se stejným rozdělením), závisí na vektoru parametrů
$\vc \theta \in \Theta \subset \Real^N$, tj.
\[
 \vc F_{\vc X}(\vc x, \vc \theta) = \prod_{i=1}^n F(x_i,\vc \theta)
\]

\xskip
Úloha: Pro známé $F$ a $n$ najít funkci (statistiku) $T_n(\vc X)$, která co nejlépe odhadne $\vc \theta$.

Při pevném $\vc \theta$ označíme $E_\theta T_n$ a $var_\theta T_n$ střední hodnotu a rozptyl statistiky $T_n$.
\end{frame}

\begin{frame}{Příklady bodových odhadů.}
 \begin{itemize}
  \item Odhadem střední hodnoty pro lib. $F$ je aritmetický průměr.
  \item Odhadem relativní četnosti $p$ (pro binomickou veličinu), je výběrová relativní četnost $n_i / n$
  \item Odhadem rozptylu pro $F \sim N(\mu, \sigma ^2)$ je výběrový rozptyl.
 \end{itemize}

\end{frame}


\begin{frame}{Požadované vlastnosti bodových odhadů}
 \begin{description}
  \item Nestrannost: Střední hodnota odhadu $E_\theta T_n$ je rovna skutečnému parametu $\vc \theta$.
  \item Konzistence: $T_n(\vc X)$ konverguje (silně) k $\vc \theta$ pro $n\to \infty$.
  \item Efficience: Odhad má nejmenší rozptyl ze všech odhadů.
  \item Asymptotická normalita.
 \end{description}

\end{frame}


\begin{frame}{Nestrannost}
\[
   E_\theta T_n = \vc \theta
\]
Př: Odhad střední hodnoty je nestranný.
\[ 
   E_\mu \overline{X} = \frac{1}{n} \sum_{i=1}^n E X_i = \mu
\]
\end{frame}

\begin{frame}{Nestranný odhad rozptylu}
Při známé střední hodnotě $\mu$:
\[
   S_n^2 = \frac{1}{n-1} \sum_{i=1}^n (X_i - \mu)^2
\]

při neznámé střední hodnotě:
\[
  S_n^2 = \frac{1}{n-1} \sum_{i=1}^n (X_i - \overline{X})^2 = \frac{1}{n-1} \sum_i X_i^2 - n\overline{X}^2
\]

\end{frame}

\begin{frame}{Odvození:}
 Nechť nezávislé $X_i$ mají střední hodnotu $\mu$ a rozptyl $\sigma^2$.
\[
   var(\overline{X}) = \frac{1}{n^2}\sum_{i=1}^n var(X_i) = \frac{\sigma^2}{n}
\]

\[
   E_{\sigma^2} S_n^2 = \frac{1}{n-1} \sum_{i=1}^n E\big((X_i-\mu) - (\overline{X} -\mu)\big)^2=
   =\frac{1}{n-1} \sum_{i=1}^n E(X_i-\mu)^2 - (\overline{X} -\mu)\big)^2
\frac{1}{n-1}(n\sigma^2 - \sigma^2)
\]
\end{frame}

\begin{frame}{Silná konzistence}
\[
   \lim_{n\to \infty} T_n(\vc X) = \theta
\]
s pravděpodobností $1$. Tj. konvergence s.j.
\xskip
Příklad:
Silný zákon velkých čísel: Průměr (odhad stř. hodnoty) konverguje s.j. ke stř. hodnotě.
\end{frame}

\begin{frame}{Eficience}
Odhad s nejmenším ``rozptylem'':
\[
\E_\theta(T_n(\vc X) - \theta)^2 \le \E_\theta(T_n^{\star}(\vc X) - \theta)^2
\]
pro každý jiný odhad $T_n$.

\end{frame}

\begin{frame}{Asymptotická normalita}
Pro odhad průměru máme:
\[
 \sqrt{n}\frac{\ol{X} - EX}/s 
\]
konverguje v distribuci k $N(0,1)$

Pro některé jiné odhady máme:
\[
 T_n(X) - \theta
\]
``konverguje'' v distribuci k $N(0,\sigma^2)$ pro vhodné $\sigma(n)$.
\end{frame}

\begin{frame}{Odhady z tříděných dat}
\end{frame}

\begin{frame}{Volby tříd}
\end{frame}

\begin{frame}{Metoda maximální věrohodnosti}
\end{frame}

\begin{frame}{Intervalové odhady}
Hledáme interval $(T_D, T_H)$ v němž přesný parametr $\theta$ leží s pstí. ({\df spolehlivostí}) $(1-\alpha)$
Významnost {\df hladina významnosti} intervalu je $\alpha$.

\begin{itemize}
 \item levostranný $P(\theta > T_D) = 1-\alpha$
 \item pravostranný $P(\theta < T_H) = 1- \alpha$
 \item oboustranný $P(\theta <T_D) = P(\theta \ge T_H) \frac{\alpha}{2}$
\end{itemize}

\end{frame}

\begin{frame}{Konkretni int. odhady !!! podrobne zpracovat, trva dlouho !!}
 
\end{frame}


\end{document}



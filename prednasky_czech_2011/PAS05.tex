
% $Header: /cvsroot/latex-beamer/latex-beamer/solutions/generic-talks/generic-ornate-15min-45min.en.tex,v 1.5 2007/01/28 20:48:23 tantau Exp $

\documentclass[smaller]{beamer}
\mode<presentation>
{
  \usetheme{Singapore}
  \usefonttheme[onlymath]{serif}
  % or ...
 %  \setbeamercovered{transparent}
  % or whatever (possibly just delete it)
}


\usepackage[czech]{babel}
% or whatever
\usepackage[utf8]{inputenc}
% or whatever
%\usepackage{times}
%\usepackage[T1]{fontenc}
% Or whatever. Note that the encoding and the font should match. If T1
% does not look nice, try deleting the line with the fontenc.


\title{PAS05 - Limitní věty}

\author{Jan B\v rezina}
\institute % (optional, but mostly needed)
{
  %\inst{2}%
  Technical University of Liberec
}


% If you wish to uncover everything in a step-wise fashion, uncomment
% the following command: 

%\beamerdefaultoverlayspecification{<+->}

% ***************************************** SYMBOLS
\def\div{{\rm div}}
\def\Lapl{\Delta}
\def\grad{\nabla}
\def\supp{{\rm supp}}
\def\dist{{\rm dist}}
%\def\chset{\mathbbm{1}}
\def\chset{1}

\def\Tr{{\rm Tr}}
\def\sgn{{\rm sgn}}
\def\to{\rightarrow}
\def\weakto{\rightharpoonup}
\def\imbed{\hookrightarrow}
\def\cimbed{\subset\subset}
\def\range{{\mathcal R}}
\def\leprox{\lesssim}
\def\argdot{{\hspace{0.18em}\cdot\hspace{0.18em}}}
\def\Distr{{\mathcal D}}
\def\calK{{\mathcal K}}
\def\FromTo{|\rightarrow}
\def\convol{\star}
\def\impl{\Rightarrow}
\DeclareMathOperator*{\esslim}{esslim}
\DeclareMathOperator*{\esssup}{ess\,sup}
\DeclareMathOperator{\ess}{ess}
\DeclareMathOperator{\osc}{osc}
\DeclareMathOperator{\curl}{curl}

%\def\Ess{{\rm ess}}
%\def\Exp{{\rm exp}}
%\def\Implies{\Longrightarrow}
%\def\Equiv{\Longleftrightarrow}
% ****************************************** GENERAL MATH NOTATION
\def\Real{{\rm\bf R}}
\def\Rd{{{\rm\bf R}^{\rm 3}}}
\def\RN{{{\rm\bf R}^N}}
\def\D{{\mathbb D}}
\def\Nnum{{\mathbb N}}
\def\Measures{{\mathcal M}}
\def\d{\,{\rm d}}               % differential
\def\sdodt{\genfrac{}{}{}{1}{\rm d}{{\rm d}t}}
\def\dodt{\genfrac{}{}{}{}{\rm d}{{\rm d}t}}

\def\vc#1{\mathbf{\boldsymbol{#1}}}     % vector
\def\tn#1{{\mathbb{#1}}}    % tensor
\def\abs#1{\lvert#1\rvert}
\def\Abs#1{\bigl\lvert#1\bigr\rvert}
\def\bigabs#1{\bigl\lvert#1\bigr\rvert}
\def\Bigabs#1{\Big\lvert#1\Big\rvert}
\def\ABS#1{\left\lvert#1\right\rvert}
\def\norm#1{\bigl\Vert#1\bigr\Vert} %norm
\def\close#1{\overline{#1}}
\def\inter#1{#1^\circ}
\def\ol#1{\overline{#1}}
\def\ul#1{\underline{#1}}
\def\eqdef{\mathrel{\mathop:}=}     % defining equivalence
\def\where{\,|\,}                    % "where" separator in set's defs
\def\timeD#1{\dot{\overline{{#1}}}}

% ******************************************* USEFULL MACROS
\def\RomanEnum{\renewcommand{\labelenumi}{\rm (\roman{enumi})}}   % enumerate by roman numbers
\def\rf#1{(\ref{#1})}                                             % ref. shortcut
\def\prtl{\partial}                                        % partial deriv.
\def\Names#1{{\scshape #1}}
\def\rem#1{{\parskip=0cm\par!! {\sl\small #1} !!}}

\def\Xint#1{\mathchoice
{\XXint\displaystyle\textstyle{#1}}%
{\XXint\textstyle\scriptstyle{#1}}%
{\XXint\scriptstyle\scriptscriptstyle{#1}}%
{\XXint\scriptscriptstyle\scriptscriptstyle{#1}}%
\!\int}
\def\XXint#1#2#3{{\setbox0=\hbox{$#1{#2#3}{\int}$}
\vcenter{\hbox{$#2#3$}}\kern-.5\wd0}}
\def\ddashint{\Xint=}
\def\dashint{\Xint-}

% ******************************************* DOCUMENT NOTATIONS
% document specific
\def\rh{\varrho}
\def\vl{{\vc{u}}}
\def\th{\vartheta}
\def\vx{\vc{x}}
\def\vX{\vc{X}}
\def\vr{\vc{r}}
\def\veta{\vc{\eta}}
\def\dx{\,\d\vx}
\def\dt{\,\d t}
\def\bulk{\zeta}
\def\cS{\close{S}}
\def\eps{\varepsilon}
\def\phi{\varphi}
\def\Bog{{\mathcal B}}
\def\Riesz{{\mathcal R}}
\def\distr{\mathcal D}
\def\Item{$\bullet$}

\def\MEtst{\mathcal T}
%***************************************************************************
\setbeamercolor{my blue}{fg=blue}
\def\blue#1{{\usebeamercolor[fg]{my blue} #1}}
\setbeamercolor{my green}{fg=green}
\def\green#1{{\usebeamercolor[fg]{my green} #1}}
\setbeamercolor{my red}{fg=red}
\def\red#1{{\usebeamercolor[fg]{my red} #1}}
\def\xskip{{\vspace{2ex}}}

\begin{document}

\begin{frame}
  \titlepage
\end{frame}

% zvýraznění definovaného pojmu
\def\df{\usebeamercolor[fg]{my red}\it}

\begin{frame}{Motivační otázky?}
\begin{itemize}
 \item Proč má řada náhodných veličin normální rozdělení?
  \pause
 \item Jak pracovat s Binomickým rozdělením pro velká $n>100$, problém výpočtu binomických koeficientů \dots
\pause
 \item Jaká je souvislost výběrového průměru a střední hodnoty?
\pause
 \item Jak moc obecné je pravidlo $3\sigma$?
\end{itemize}

\end{frame}


\section{Limitní věty}
\begin{frame}{konvergence náhodných veličin}
Uvažujme posloupnost náhodných veličin $X_n$. A náhodnou veličinu $X$. Říkáme, že
\begin{itemize}
 \item $X_n$ konverguje {\df skoro jistě} k $X$, pokud pravděpodobnost konvergence je $1$, tj.
 \[
    P\{ \omega\in \Omega | X_n(\omega) \to X(\omega) \} = 1
 \]
 zkráceně: $X_n\to X$ s.j.
 \item $X_n$ konverguje {\df podle pravděpodobnosti}, pokud
 \[
   P\{\omega\in\Omega: \abs{X_n(\omega) -X(\omega)}> \eps\} \to  \quad \forall\, \eps> 0
 \]
 zkráceně: $X_n \stackrel{P}{\to} X$.
 \item $X_n$ konverguje {\df podle středu}, pokud 
 \[
    E(X_n - X)^2 \to 0
 \]
 \item $X_n$ konverguje {\df v distribuci}, pokud 
 \[
   \lim_{n\to\infty} F_n(x) = F(x)\quad \forall\, x\in\Real
 \]
\end{itemize}
\end{frame}

\begin{frame}{Vztahy mezi konvergencemi}
 \begin{itemize}
  \item ``skoro jistě``, ``podle psti.`` a ``podle středu'' implikuje ''v distribuci``
  \item ''skoro jistě`` a ''podle středu`` implikuje '' podle psti.``
 \end{itemize}

\end{frame}

\begin{frame}{Markovova nerovnost}
\begin{theorem}
 Pokud $X$ je náhodná veličina, pak $P(\abs{X}\ge a) \le \frac{E\abs{X}}{a}$
\end{theorem}
Důkaz:\\
$\abs{X}\ge aI_{\abs{X}\ge a} $, kde $I_M=1$ pokud nastal jev $M$ (indikátorová NV)\\
aplikujeme střední hodnotu:
\[
  E(\abs{X}) \ge E(aI_{\abs{X}\ge a}) = a P(\abs{X}\ge a)
\]
\end{frame}

\begin{frame}{Čebyševova nerovnost}
\[
   P\{\abs{X-EX}\ge a \sigma(X)\} \le \frac{1}{a^2}
\]
pro unimodální: je LHS $\le \frac{4}{9a^2}$ (Vysochanskiï–Petunin inequality), pravidlo $3$ sigma 

\end{frame}



\begin{frame}{slabý Zákon velkých čísel }
Nechť $X_n$ je posloupnost nezávislých n.v. se stejnými stř. hodnotami $EX_n=\mu$ pak
pro n.v.
\[
  \overline{X_n}:= \frac{1}{n} \sum_{i=1}^n X_i
\]
platí
\[
 \overline{X_n} \stackrel{P}{\to} \mu
\]
\end{frame}

\begin{frame}{Důkaz:}
Důkaz pokud předpokládáme omezené rozptyly: $DX_n \le C$.
\[
  E\overline{X_n} = \mu
\]
Pro rozptyl součtu platí:
\[
  var(X+Y) = var(X) + var(Y) + 2cov(X,Y)
\]
a pro nezávislé veličiny je $cov(X,Y) = 0$, tady: 

\[
D\overline{X_n} = 1/{n^2} \sum_i var(X_i)\le \frac{C}{n} )
\]
Markovova nerovnost: 
\[
  P(\abs{\overline{X_n} - \mu} \ge \eps ) = P(\abs{\overline{X_n} - \mu}^2 \ge \eps^2 ) \le \frac{C}{n\eps^2} \to 0
\]
\end{frame}



\begin{frame}{silný Zákon velkých čísel}
Nechť $X_n$ je posloupnost nezávislých n.v. se stejným rozdělením a konečnou střední hodnotou $EX_i = \mu$ pak
platí
\[
 \overline{X_n} \to \mu \quad \text{skoro jistě.}
\]
\end{frame}

\begin{frame}{Centrální limitní věta}
Nechť $X_n$ je posloupnost nezávislých stejně rozdělených n.v. s konečnou $EX_i =\mu$ a $DX_i =\sigma^2$, pak 
\[
  \sqrt{n}(\ol{X}_n -\mu)\to N(0, \sigma^2)\quad\text{v distribuci}
\]
nebo-li
\[
  Z_n=\sqrt{n}\frac{\ol{X}_n-\mu}{\sigma} \to N(0,1)\quad\text{v distribuci}
\]

Platí i pro různá rozdělení, ale je třeba předpokládat jisté vlastnosti vyšších momentů.

Zjednodušeně: 
\[
 \sum_{i=1}^{n} X_i  ''\sim `` N(n\mu, n\sigma^2)
\]
\[
 \ol{X}_n ''\sim`` N(\mu, \frac{\sigma^2}{n})
\]


\end{frame}

\begin{frame}{Moivrova-Laplaceova věta}
\[
 Bi(n,p) ''\sim`` N(np, np(1-p) )
\]
pokud je $n$ dost veliké resp.
\[
 np(1-p) \ge 9
\]
nebo
\[
 np > 5\quad \text{a} \quad n(1-p) > 5
\]


\end{frame}

\end{document}



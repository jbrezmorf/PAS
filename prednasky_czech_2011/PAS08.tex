
% $Header: /cvsroot/latex-beamer/latex-beamer/solutions/generic-talks/generic-ornate-15min-45min.en.tex,v 1.5 2007/01/28 20:48:23 tantau Exp $

\documentclass[smaller]{beamer}
\mode<presentation>
{
  \usetheme{Singapore}
  \usefonttheme[onlymath]{serif}
  % or ...
 %  \setbeamercovered{transparent}
  % or whatever (possibly just delete it)
}

\usepackage{amssymb}
\usepackage[czech]{babel}
% or whatever
\usepackage[utf8]{inputenc}
% or whatever
%\usepackage{times}
%\usepackage[T1]{fontenc}
% Or whatever. Note that the encoding and the font should match. If T1
% does not look nice, try deleting the line with the fontenc.


\title{PAS08 -- Dvouvýběrové testy}

\author{Jan B\v rezina}
\institute % (optional, but mostly needed)
{
  %\inst{2}%
  Technical University of Liberec
}


% If you wish to uncover everything in a step-wise fashion, uncomment
% the following command: 

%\beamerdefaultoverlayspecification{<+->}

% ***************************************** SYMBOLS
\def\div{{\rm div}}
\def\Lapl{\Delta}
\def\grad{\nabla}
\def\supp{{\rm supp}}
\def\dist{{\rm dist}}
%\def\chset{\mathbbm{1}}
\def\chset{1}

\def\Tr{{\rm Tr}}
\def\sgn{{\rm sgn}}
\def\to{\rightarrow}
\def\weakto{\rightharpoonup}
\def\imbed{\hookrightarrow}
\def\cimbed{\subset\subset}
\def\range{{\mathcal R}}
\def\leprox{\lesssim}
\def\argdot{{\hspace{0.18em}\cdot\hspace{0.18em}}}
\def\Distr{{\mathcal D}}
\def\calK{{\mathcal K}}
\def\FromTo{|\rightarrow}
\def\convol{\star}
\def\impl{\Rightarrow}
\DeclareMathOperator*{\esslim}{esslim}
\DeclareMathOperator*{\esssup}{ess\,sup}
\DeclareMathOperator{\ess}{ess}
\DeclareMathOperator{\osc}{osc}
\DeclareMathOperator{\curl}{curl}

%\def\Ess{{\rm ess}}
%\def\Exp{{\rm exp}}
%\def\Implies{\Longrightarrow}
%\def\Equiv{\Longleftrightarrow}
% ****************************************** GENERAL MATH NOTATION
\def\Real{{\rm\bf R}}
\def\Rd{{{\rm\bf R}^{\rm 3}}}
\def\RN{{{\rm\bf R}^N}}
\def\D{{\mathbb D}}
\def\Nnum{{\mathbb N}}
\def\Measures{{\mathcal M}}
\def\d{\,{\rm d}}               % differential
\def\sdodt{\genfrac{}{}{}{1}{\rm d}{{\rm d}t}}
\def\dodt{\genfrac{}{}{}{}{\rm d}{{\rm d}t}}

\def\vc#1{\mathbf{\boldsymbol{#1}}}     % vector
\def\tn#1{{\mathbb{#1}}}    % tensor
\def\abs#1{\lvert#1\rvert}
\def\Abs#1{\bigl\lvert#1\bigr\rvert}
\def\bigabs#1{\bigl\lvert#1\bigr\rvert}
\def\Bigabs#1{\Big\lvert#1\Big\rvert}
\def\ABS#1{\left\lvert#1\right\rvert}
\def\norm#1{\bigl\Vert#1\bigr\Vert} %norm
\def\close#1{\overline{#1}}
\def\inter#1{#1^\circ}
\def\ol#1{\overline{#1}}
\def\ul#1{\underline{#1}}
\def\eqdef{\mathrel{\mathop:}=}     % defining equivalence
\def\where{\,|\,}                    % "where" separator in set's defs
\def\timeD#1{\dot{\overline{{#1}}}}

% ******************************************* USEFULL MACROS
\def\RomanEnum{\renewcommand{\labelenumi}{\rm (\roman{enumi})}}   % enumerate by roman numbers
\def\rf#1{(\ref{#1})}                                             % ref. shortcut
\def\prtl{\partial}                                        % partial deriv.
\def\Names#1{{\scshape #1}}
\def\rem#1{{\parskip=0cm\par!! {\sl\small #1} !!}}

\def\Xint#1{\mathchoice
{\XXint\displaystyle\textstyle{#1}}%
{\XXint\textstyle\scriptstyle{#1}}%
{\XXint\scriptstyle\scriptscriptstyle{#1}}%
{\XXint\scriptscriptstyle\scriptscriptstyle{#1}}%
\!\int}
\def\XXint#1#2#3{{\setbox0=\hbox{$#1{#2#3}{\int}$}
\vcenter{\hbox{$#2#3$}}\kern-.5\wd0}}
\def\ddashint{\Xint=}
\def\dashint{\Xint-}

% ******************************************* DOCUMENT NOTATIONS
% document specific
\def\rh{\varrho}
\def\vl{{\vc{u}}}
\def\th{\vartheta}
\def\vx{\vc{x}}
\def\vX{\vc{X}}
\def\vr{\vc{r}}
\def\veta{\vc{\eta}}
\def\dx{\,\d\vx}
\def\dt{\,\d t}
\def\bulk{\zeta}
\def\cS{\close{S}}
\def\eps{\varepsilon}
\def\phi{\varphi}
\def\Bog{{\mathcal B}}
\def\Riesz{{\mathcal R}}
\def\distr{\mathcal D}
\def\Item{$\bullet$}

\def\MEtst{\mathcal T}
%***************************************************************************
\setbeamercolor{my blue}{fg=blue}
\def\blue#1{{\usebeamercolor[fg]{my blue} #1}}
\setbeamercolor{my green}{fg=green}
\def\green#1{{\usebeamercolor[fg]{my green} #1}}
\setbeamercolor{my red}{fg=red}
\def\red#1{{\usebeamercolor[fg]{my red} #1}}
\def\xskip{{\vspace{2ex}}}
\def\E{\vc{\mathsf{E}}}

\begin{document}

\begin{frame}
  \titlepage
\end{frame}

% zvýraznění definovaného pojmu
\def\df{\usebeamercolor[fg]{my red}\it}
\begin{frame}{Dvouvýběrový $t$-test (stejné rozptyly)}
předpoklady:\\
$X_1, \dots ,X_m$ výběr z $N(\mu_1,\sigma_1^2)$,\\
$Y_1, \dots ,Y_n$ výběr z $N(\mu_2,\sigma_2^2)$,\\
\blue{$\sigma_1 = \sigma_2$}, test není příliš citlivý na porušení tohoto předpokladu

\xskip
$H_0: \mu_1 - \mu_2  = \delta$ vs. alternativy $H_A: \mu_1 - \mu_2  \lesseqqgtr \delta$

\xskip
za hypotézy $H_0$ má $T$ Studentovo rozdělení s $n+m-2$ stupni volnosti:
\[
 T = \frac{\ol{X} - \ol{Y} - \delta}{s_p\sqrt{1/n +1/m}} \sim t_{n+m-2}
\]
kde
\[
   s_p = \sqrt{\frac{(n-1)s_X + (m-1)s_Y}{n+m-2}}
\]
a $\ol{X}, \ol{Y}$ jsou průměry a $s_X, s_Y$ výběrové směrodatné odchylky.

$p$-value pak počítáme pomocí distribuční funkce Studentova rozdělení: $F_{t_{n+m-2}}(T)$
\end{frame}

\begin{frame}{Dvouvýběrový $t$-test (různé rozptyly)}
\dots případ $\sigma_1 \ne \sigma_2$ (výrazný rozdíl),
stejné $H_0$, $H_A$

\blue{Cochranův-Coxův test}
    \[
      v_X = \frac{s_X^2}{m},\qquad v_Y = \frac{s_Y^2}{n},\qquad S = \sqrt{v_X + v_Y}
    \]
    \[
      T^\star = \frac{\ol{X} - \ol{Y} -\delta}{S},\qquad t^\star = \frac{v_X t_{m-1}(\alpha) + v_Y t_{n-1}(\alpha)}{v_X +v_Y}
    \]
    $H_0$ se zamítá ve prospěch $H_A: \mu_1 -\mu_2 \ne\delta$ pokud $\abs{T^\star} \ge t^\star$

\blue{Aspinové-Welchův test}
   veličina $T^\star$ má přibližně rozdělení $t_f$ pro 
   \[
      f=\frac{S^4}{\frac{v_X^2}{m-1} + \frac{v_Y^2}{n-1}}
   \]

\blue{Satterthwaiteův test}
   veličina $T^\star$ má přibližně rozdělení $t_h$ pro 
   \[
      h=\frac{S^4}{\frac{v_X^2}{m+1} + \frac{v_Y^2}{n+1}} -2
   \]
Poslední dva testy lze provést pomocí $p$-value.\\
Je třeba počítat s neceločíselnými stupni volnosti.
\end{frame}

\begin{frame}{Příklad}
Na $8$ polích se zkoušel nový způsob hnojení (s výnosy $X_i$),\\ 
na $5$ polích starý způsob (výnosy $Y_i$).

\[
 m=8,\quad n=5,\quad \ol{X}=5.387,\quad \ol{Y}=4.7,\quad s_X^2=0.2698,\quad s_Y^2=0.24
\]
Studentův test, $H_A: \mu_X\ne \mu_Y$: \\
$\abs{T}=2.37$ větší než kritická hodnota $t_{11}(0.025) =2.201$, zamítáme, $p_{val} = 0.037 < 0.5$

\xskip
Studentův test,  $H_A: \mu_X > \mu_Y$: \\
$T=2.37$ větší než kritická hodnota $t_{11}(0.05) =1.796$, zamítáme, $p_{val} = 0.019 < 0.5$
\end{frame}

\begin{frame}{Příklad\dots aplikace testů  při odlišných rozptylech}
Cochran $H_A:  \mu_X\ne \mu_Y$:
$T^{\star} = 2.4$, $t^{\star} = 2.6$, nezamítá $H_0$
\xskip
Aspinové-Welchův $f=9.044$, $t_f(0.025) = 2.26$, zamítá $H_0$, $p_{val} = 0.42$

\xskip
Satterthwaiteův $h=11.086$, $t_h(0.025) =2.2$, zamítá $H_0$, $p_{val} = 0.037 $

\end{frame}




\begin{frame}{test shody rozptylů (F-test)}
$X_1, \dots ,X_m$ výběr z $N(\mu_1,\sigma_1^2)$,\\
$Y_1, \dots ,Y_n$ výběr z $N(\mu_2,\sigma_2^2)$
$s_X \ge s_Y$ ( pro jednodušší provedení testu)

\[
  T=\frac{s_X^2}{s_Y^2} \sim F_{m-1,n-1},
\]
kde $F_{M,N}$ je Fisherovo-Snedekorovo rozdělení.

$H_0:\sigma_1 = \sigma_2$ zamítneme ve prospěch $H_A: \sigma_1\ne\sigma_2$ pokud $T$ je mimo interval
daný hodnotami:
\[
 F_{m-1,n-1}\Big(1-\frac{\alpha}{2}\Big) = \frac{1}{F_{n-1,m-1}\big(\frac{\alpha}{2}\big)}
 \quad\text{a}\quad
 F_{n-1,m-1}\Big(\frac{\alpha}{2}\Big)
\]
resp. (za předpokladu $s_X \ge s_Y$) pokud
\[
  T \ge F_{n-1,m-1}\Big(\frac{\alpha}{2}\Big)
\]
\end{frame}

\begin{frame}{Příklad}
viz. zadání o hnojení: 
\[
 m=8,\quad n=5,\quad s_X^2=0.2698,\quad s_Y^2=0.24
\]
$T = 1.124\quad <\quad F_{7,4}(0.025) = 9.07$,\\
test nezamítá shodu rozptylů, $p_{val}=0.52$
 
\end{frame}

\begin{frame}{Vztah binomického a Fisherova rozdělení}
Distribuční funkci $Bi(n,p)$ je možno počítat z Fisherova rozdělení $F_{2(n-s),2(s+1)}$:
\[
 F_{Bi}(s) = \sum_{i=0}^s \binom{n}{i} p^i (1-p)^i = F^\star_{2(n-s),2(s+1)}\Big(\frac{(s+1)(1-p)}{p(n-s)}\Big) 
\]
Intervalový odhad $(L,H)$ pro relativní četnost $p$ při výběrové četnosti $Y\sim Bi(n,p)$ (tj. při pokusu nastalo $Y$ příznivých případů z $n$)
\[
 L = \frac{Y}{Y +(n-Y+1)F_{2(n-Y+1),2Y}(\alpha / 2)},
\]
\[
 H = \frac{(Y+1)F_{2(Y+1),2(n-Y)}(\alpha / 2)}{n-Y +(Y+1)F_{2(Y+1),2(n-Y)}(\alpha / 2)}.
\]

\end{frame}


\begin{frame}{test homogenity binomických rozdělení, postup A}
$H_0: p_X = p_Y$, 

výběrové relativní četnosti:
\[
 x=\frac{X}{m}\qquad y=\frac{Y}{n}
\]
statistika (aproximujeme relativní četnosti zvlášť):
\[
 U_a = \frac{x-y}{\sqrt{\frac{x(1-x)}{m} + \frac{y(1-y)}{n}}} \quad \sim N(0,1)
\]

$H_0$ se zamítá pokud $U_a \ge u(\alpha / 2)$ 
\end{frame}

\begin{frame}{test homogenity binomických rozdělení, postup B}
Aproximace s polečné reletivní četnosti $p_X = p_Y$ (předpokládáme $H_0$):
\[
 z = \frac{X + Y}{m+n}
\]
staistika:
\[
 U_b = \frac{x-y}{\sqrt{z(1-z)\big(\frac{1}{m} + \frac{1}{n}\big)}} \quad \sim N(0,1)
\]

$H_0$ se zamítá pokud $U_b \ge u(\alpha / 2)$. 
\end{frame}


\begin{frame}{testy shody distribučních funkcí Kolmogorov-Smirnov}
$X_1, \dots ,X_m$ výběr ze spojitého rozdělení $F$,\\
$Y_1, \dots ,Y_n$ výběr ze spojitého rozdělení $G$,\\
nezávislé veličiny

\xskip
$H_0: F=G$ oproti $H_A: F\ne G$\\
Empirické distribuční funkce $F_m$, $G_n$

\[
 D_{m,n} = \sup_{x}\abs{F_m(x) - G_n(x)}
\]

$H_0$ se zamítá pokud $D_{m,n}$ je větší než kritická hodnota $D_{m,n}(\alpha)$. Pro malé hodnoty $m,n$ viz. tabulky, Ostravská skripta.
Pro větší hodnoty aproximace:
\[
 D^\star_{m,n}(\alpha) = \sqrt{\frac{1}{2M} \ln\frac{2}{\alpha}}, \qquad M = \frac{mn}{m+n}
\]
\end{frame}



\documentclass[4pt]{article}
\usepackage[a5paper]{geometry}
\usepackage{pgfpages} 
\pgfpagesuselayout{2 on 1}[a4paper,landscape,border shrink=5mm]

%\usepackage[active]{srcltx}
%\usepackage[czech]{babel}
%\usepackage[utf8]{inputenc}
\usepackage{color}

\usepackage{amsmath}
\usepackage{amsfonts}
\usepackage{amssymb}
\usepackage{amsthm}
\usepackage{listings}
\usepackage{textcomp}
%
\newtheorem{theorem}{Věta}[section]
\newtheorem{proposition}[theorem]{Tvrzení}
\newtheorem{definition}[theorem]{Definice}
\newtheorem{remark}[theorem]{Poznámka}
\newtheorem{lemma}[theorem]{Lemma}
\newtheorem{corollary}[theorem]{Důsledek}
\newtheorem{exercise}[theorem]{Cvičení}

%\numberwithin{equation}{document}
%
\def\div{{\rm div}}
\def\Lapl{\Delta}
\def\grad{\nabla}
\def\supp{{\rm supp}}
\def\dist{{\rm dist}}
%\def\chset{\mathbbm{1}}
\def\chset{1}
%
\def\Tr{{\rm Tr}}
\def\to{\rightarrow}
\def\weakto{\rightharpoonup}
\def\imbed{\hookrightarrow}
\def\cimbed{\subset\subset}
\def\range{{\mathcal R}}
\def\leprox{\lesssim}
\def\argdot{{\hspace{0.18em}\cdot\hspace{0.18em}}}
\def\Distr{{\mathcal D}}
\def\calK{{\mathcal K}}
\def\FromTo{|\rightarrow}
\def\convol{\star}
\def\impl{\Rightarrow}
\DeclareMathOperator*{\esslim}{esslim}
\DeclareMathOperator*{\esssup}{ess\,supp}
\DeclareMathOperator{\ess}{ess}
\DeclareMathOperator{\osc}{osc}
\DeclareMathOperator{\curl}{curl}
%
%\def\Ess{{\rm ess}}
%\def\Exp{{\rm exp}}
%\def\Implies{\Longrightarrow}
%\def\Equiv{\Longleftrightarrow}
% ****************************************** GENERAL MATH NOTATION
\def\Real{{\rm\bf R}}
\def\Rd{{{\rm\bf R}^{\rm 3}}}
\def\RN{{{\rm\bf R}^N}}
\def\D{{\mathbb D}}
\def\Nnum{{\mathbb N}}
\def\Measures{{\mathcal M}}
\def\d{\,{\rm d}}               % differential
\def\sdodt{\genfrac{}{}{}{1}{\rm d}{{\rm d}t}}
\def\dodt{\genfrac{}{}{}{}{\rm d}{{\rm d}t}}
\def\dx{\d x}
\def\Lin{\mathit{Lin}}
%
\def\vc#1{\mathbf{\boldsymbol{#1}}}     % vector
\def\tn#1{{\mathbb{#1}}}    % tensor
\def\abs#1{\lvert#1\rvert}
\def\Abs#1{\bigl\lvert#1\bigr\rvert}
\def\bigabs#1{\bigl\lvert#1\bigr\rvert}
\def\Bigabs#1{\Big\lvert#1\Big\rvert}
\def\ABS#1{\left\lvert#1\right\rvert}
\def\norm#1{\bigl\Vert#1\bigr\Vert} %norm
\def\close#1{\overline{#1}}
\def\inter#1{#1^\circ}
\def\eqdef{\mathrel{\mathop:}=}     % defining equivalence
\def\where{\,|\,}                    % "where" separator in set's defs
\def\timeD#1{\dot{\overline{{#1}}}}
%
% ******************************************* USEFULL MACROS
\def\RomanEnum{\renewcommand{\labelenumi}{\rm (\roman{enumi})}}   % enumerate by roman numbers
\def\rf#1{(\ref{#1})}                                             % ref. shortcut
\def\prtl{\partial}                                        % partial deriv.
\def\Names#1{{\scshape #1}}
\def\rem#1{{\parskip=0cm\par!! {\sl\small #1} !!}}
%\def\reseni#1{\par{\bf Řešení:}#1}
%
%
% ******************************************* DOCUMENT NOTATIONS
% document specific
%***************************************************************************
%


%\def\result#1{{\color{green} #1}}
\def\result#1{}
%\def\answer#1{{\color{red} #1}}
\def\answer#1{}

\addtolength{\textwidth}{3cm}
\addtolength{\textheight}{4.5cm}
\addtolength{\topmargin}{-2cm}
\addtolength{\hoffset}{-1.5cm}
\begin{document}
%%%%%%%%%%%%%%%%%%%%%%%%%%%%%%%%%%%%%%%%%%%%%%%%%%%%%%%%%%%%%%%%%%%%%%%%%%%%%%%%%%%%%%%%%%%%%%%%%%%%%%%
\section{Exercise 1 (R and explratory statistics)}
%
% 1) vector matrix operations
% 2) read input data (datafile with 2 random discrete and 2 scalar data, one random)
% 3) basic visuzlization 
% scatter plot - cars, 
% histogram - islands
% boxplot - InsectSprays, OrchardSprays, ToothGrowth
% 4) generating random data (rnorm)
% 5) data packages
% 4) basic statistics (mean, var, sd, summary, quantile, ecdf, IQR, mad, curtoisis, skewness (own implementation))
% 5) further poloting: matplot, pairs, cleveland plot (lattice)
%
% Z praxe: stihly se jen prvni dve sekce, podrobneji rozepsat, spis primo prikazy ktere mohou zadat a zjistit co to dela
% 

\subsection{Vectors and Matrices}

\begin{enumerate}
 \item assign $2$ to \verb'x', compute $\exp(-x^2)$ and assign result to \verb'gauss_x'
 \item create a vector \verb'vx' with elements $-2,\;1,\;1.3,\;5.4$;
 \item compute length, sum, and mean for \verb'vx', compute \verb'gauss_vx'
 \item print even elements, print elements greater then $1$; print \verb'vx[0]' and \verb'vx[5]'
 \item get help for \verb'seq', create vector from interval $-1,1$ with step $0.1$
 \item get help for \verb'factor', create factor from vector \verb'c(1,3,2,1,2)', use labels A, B, C.
 \item create any (regular) matrix $2\times 2$ and find its inverse; check its validity
\end{enumerate}

\subsection{Dataframes and plots}
\begin{enumerate}
 \item get help for dataset \verb'chickwts', rename \verb'feed' to \verb'diet'
 \item unstack the dataset to list of named vectors, stack it back
 \item make stripchart, use different colors, filled points, add title
 % striptchart(chickwts$weigth~chickwts$feed)
 % stripchart( chickwts$weight ~ chickwts$feed, las=2, col=rev(c('yellow','brown','red','blue', 'orange', 'black')), method="stack", pch=16)
 \item make boxplot for individual groups, horizontal
 % boxplot( weight ~ feed, data=chickwts, col=rev(c('yellow','brown','red','blue', 'orange', 'gray')), horizontal=T, names=levels(chickwts$feed))
 % axis(2, labels=levels(chickwts$feed), at=1:6, las=2)
 \item put both into same plot
 \item use read.table to read in file \verb'cars.txt' as \verb'cars_df'
 \item \verb'attach' it; make scatterplot with plot command
 \item try \verb'example(cars)'
\end{enumerate}

\subsection{Random generators}
\begin{enumerate}
 \item use \verb'sample' to simulate $100$ dice throws
 \item use \verb'rnorm' to get histogram of $1000$ values with normal dist.
 \item use \verb'ecdf' to previous data 
 \item add plot of true CDF using \verb'pnorm'
 \item use \verb'qqnorm'
 \item use \verb'qqplot' directly to get same result
\end{enumerate}
\pagebreak

\subsection{More plotting}



\pagebreak
%%%%%%%%%%%%%%%%%%%%%%%%%%%%%%%%%%%%%%%%%%%%%%%%%%%%%%%%%%%%%%%%%%%%%%%%%%%%%%%%%%%%%%%%%%%%%%%%%%%%%%%
\section{Exercise 2}
%
% Classical probability: 
% 

\subsection{Dice sequence}
What is probability to throw a sequence when throwing 6 dices? \result{$0.0154$}
Check answer numericaly using: 
\begin{verbatim}
gen=rmultinom(100000, 6, ...)
is_seq=function(vec) prod(vec - ...)
sum(apply(gen,2, is_seq)) ...
\end{verbatim}


\subsection{Selection with replacement}
   From 10 keys there are 3 with defect. If you take 5 randomly, what is the 
   probability that they include 2 keys with defect? \result{$0.41666$} Give a formula in terms of comination numbers and evaluate in R
   using: \verb'choose'

   Check result with \verb'dhyper(2,3,7,5)'. Try \verb'barplot(dhyper(0:5,3,7,5))'.

\begin{verbatim}  
  key_set=c(rep(1,3),rep(0,7))
  w_repl=replicate(10000,sum(sample(key_set, 5, replace=T)))
  wo_repl=replicate(10000,sum(sample(key_set, 5, replace=F)))
  
  hist(w_repl)
  hist(wo_repl)
  points(0:5, 10000* dhyper(0:5, 5, 5, 5))
  points(0:5, 10000* dbinom(0:5, 5, 0.5))
\end{verbatim}
% TODO: good candidate to use Claveland plot

\subsection{Random Events as sets}
What is the probability that a positive integer selected at random from
the set of positive integers not exceeding $100$ is divisible by either $2$ or $5$?

\answer{
\[
 P(A \vee B) = P(A) + P(B) - P(A.B) = 0.5 + 0.2 - 0.1 = 0.6 
\]
}



\subsection{Conditional probability}
Three machines has probabilities $0.8$, $0.7$, $0.4$ to produce a right component. 
What is probability that you obtain exactly one right component? What are probabilities
that this component was produced by 1., 2., 3. machine. 
\answer{
\[
    P(A)=P(A_1'A_2A_3 + A_1 A_2'A_3 + A_1A_2 A_3')
\]
]
}
\result{
\[
        P(A) = 0.252,\quad 
        P(A1/A) = 0.571\quad
        P(A2/A) = 0.333\quad
        P(A3/A) = 0.095
\]        
}


\subsection{Total probability}
There are $7$ white, $9$ black, and $11$ green balls in the box. We draw one ball, replace it into the box and add another $m$ balls of the same color. 
What is the probablity to draw a black ball in the second draw?
\answer{
\[
    \frac{7}{27}\cdot\frac{9}{27+m}+\frac{9}{27}\cdot\frac{9+m}{27+m}+\frac{11}{27}\cdot\frac{9}{27+m} = \frac{1}{3}
\]    
}

\subsection{Bayes formula}
\[
   P(A|B) = \frac{P(B | A) P(A)}{P(B)} 
\]
A retail store carries a product that is supplied by three manufactures, A (30\%), B (20\%), and C(50\%). 
Percentage of defective products of these supliers are: 2\%, 3\%, and 5\%, respectively.
\begin{enumerate}
 \item If a product is randomly selected from this store, what is the probability that it is
defective? \answer{ $P(D|A)P(A) + \dots=0.3*0.02 +0.2*0.03 + +0.5*0.05 = 0.037$ }
 \item If a defective product is found what is the probability that it was from A? Same for B and C.
 \answer{ 
 $P(A|D)=P(D|A)P(A)/P(D)=0.162$, $0.162$, $0.676$
 }
\end{enumerate}


\pagebreak

%%%%%%%%%%%%%%%%%%%%%%%%%%%%%%%%%%%%%%%%%%%%%%%%%%%%%%%%%%%%%%%%%%%%%%%%%%%%%%%%%%%%%%%%%%%%%%%%%%%%%%%
\section{Excercise 3}
% Priklady na pouziti konkretnich rozdeleni, nahodne veliciny, nahodne vektory
\subsection{Uniform distribution}
Compute expectance and variance of uniform distribution over interval $[a,b]$.
Draw graf of its probability density function and its cumulative distribution function.

\subsection{Discrete variable}
Consider disrete random variable $X$ getting values $x_i=0,\dots,4$ with probabilities $p_i=n_i/16$, where
$n_i=1,4,6,4,1$.

\begin{enumerate}
 \item Draw its distribution function (in R). Ingredients: \verb'cumsum' and \verb'stepfun'.
 \item Plot distribution function, determine median, and quartiles.
 \item Compute expectation and variance without using \verb'mean' and \verb'var' functions.
\end{enumerate}

\subsection{Choose distribution 1}
A complex electrical instrument consists of 1000 components. Probability of a defect of one component over one year is $p=0.001$
and it is independent of defect of other components.
\begin{enumerate}
 \item Choose suitable theoretical discrete random variable to describe number of defects over one year. 
 \item Determine probability of two defects over one year.
 \item Determine probability of at least two defects over one year.
\end{enumerate}

\subsection{Choose distribution 2}
Products are sorted into two classes according its quality. Probabily of a first class product is 70\% and
probability of the second class product is 30\%. Determine probability to get 15 products of the second class 
in a set of 20 products.

\subsection{Choose distribution 3}
% Moc cviceni 8/3
% not clear why it works, see R_experiments



\pagebreak
nic
\pagebreak
%%%%%%%%%%%%%%%%%%%%%%%%%%%%%%%%%%%%%%%%%%%%%%%%%%%%%%%%%%%%%%%%%%%%%%%%%%%%%%%%%%%%%%%%%%%%%%%%%%%%%%%
\section{Exercise 4}

\subsection{Strong law of large numbers}
Numericaly confirm validity of the large numbers law for the average of an vector of alternative random variables.
\[
   \close{X}_n = \frac{1}{n} \sum_{i=1}{n} X_i, \qquad X_i \sim Alt(p)
\]
Plot realization of random sequence: $A_n = \close{X}_n - p$ for $n\approx 1.01^i$, $i=2,\dots,1000$.

Code template:
\begin{lstlisting}[language=r]
# average from alternative distr.
p_param=0.5
avg_alt=fn=function(n) ( r.....(1,n,p_param)/n - p_param )

# average from normal distr.
mu_par=-3.0
var_par=100000.0
avg_norm=function(n) {
  vec=rnorm(n,...,...)
  return( sum(vec)/n - ... )
} 

n_vec=round(1.1^(2:100))
avg_vec=sapply(n_vec, ...)
plot(n_vec,avg_vec, type='l', log='x')
\end{lstlisting}

Tasks:
\begin{enumerate}
 \item Complete the code to get convergence plot for averages of $n$ alternative variables with $p=0.4$.
 \item Try to set $p=0.01$. Try to change range of \verb'n_vec'.
 \item Use \verb'avg_norm' function to get similar plot for averages of normal distribution. Does speed of convergence depend on the parameters of the normal distribution?
\end{enumerate}

\subsection{Aproximating binomial variable}
Compute \verb'dbinom(3,100,0.37)' withou using \verb'dbinom' command. Use approximation:
\[
   Bi(n,p) \approx N(np,np(1-p).
\]
Read help for \verb'dnorm'.

Complete following code to get comparison of approximation with true binomial distr.
\begin{lstlisting}[language=r]
...
plot(x,dbinom(x,n,p), type="l", col="red")
lines(x,dnorm(x, ..., ... ,col="blue")
\end{lstlisting}



\subsection{Central limit theorem}
\begin{lstlisting}[language=r]

clt_fn=function(n)  {
  mu=...
  sigma=...
  avg=sum(runif(n))/n
  return (...*(avg - mu)/sigma)
}  

N=4
n_vec=rep(N, 10000)
avg_vec=sapply(n_vec, clt_fn)
hist(avg_vec, breaks=50,freq=FALSE)

x=seq(-4,4,0.01)
y=dnorm(x)
lines(x,y, col="red")
\end{lstlisting}
Tasks:
\begin{enumerate}
 \item Try to change $N$ from $1$ up to $10$. For which $N$ it ssems that histogram match normal distribution?
 \item Try to tune other parameters to discover difference even for large $N$.
 \item Try to use \verb'qqnorm(avg_vec)' to detect deviation from normality.
 \item Compute and plot difference between ECDF and \verb'pnorm'.
\end{enumerate}

\pagebreak

%%%%%%%%%%%%%%%%%%%%%%%%%%%%%%%%%%%%%%%%%%%%%%%%%%%%%%%%%%%%%%%%%%%%%%%%%%%%%%%%%%%%%%%%%%%%%%%%%%%%%%%

\section{Exercise 5}

\subsection{Mean value tests and estimates (Z01)}
Measurement of heat conductivity of a wall gives results:
\[
    0.62, 0.64, 0.57,0.61, 0.59, 0.57, 0.62, 0.59
\]
Test if the coefficient satisfies norm $0.6$.

\begin{enumerate}
 \item compute average and sample standard deviance
 \item compute $t$-statistics
 \item compute critical quantile
 \item make two-side and one side classical test
 \item make p-value tests
 \item use R function \verb'test.t' to preform both tests and determine $95\%$ interval estimates
\end{enumerate}

\subsection{Proportion test (Z05)}
Governor got 60\% votes in last poll. In survay counting 100 citizens,
47 would vote for him in upcomming poll. Can we expect same result in upcomming poll?
\begin{itemize}
 \item use R \verb'prop.test'
 \item use approximation, discuss its validity
\end{itemize}


\subsection{Variance test (Z04)}
Volf priklady pro AMA
Z04 - proportion test, different size
Z06 - odhad stredni hodnoty, velikost souboru
Z02 - parovy t-test
Z05 - proportion test (asymptoticky i pomoci R)
A11 - chi-test
A12 

A01 - test na Poisonovo rozdeleni
A05 !! - test medianu

\pagebreak

%%%%%%%%%%%%%%%%%%%%%%%%%%%%%%%%%%%%%%%%%%%%%%%%%%%%%%%%%%%%%%%%%%%%%%%%%%%%%%%%%%%%%%%%%%%%%%%%%%%%%%%
\noindent
\pagebreak

%%%%%%%%%%%%%%%%%%%%%%%%%%%%%%%%%%%%%%%%%%%%%%%%%%%%%%%%%%%%%%%%%%%%%%%%%%%%%%%%%%%%%%%%%%%%%%%%%%%%%%%

\section{Exercise 6}

\subsection{Two sample $t$-test (A13)}
download datafile: \verb'http://atrey.karlin.mff.cuni.cz/~morf/vyuka/pas/cvika/cvika_6_kalcit.dat'

\begin{itemize}
 \item make box plot of data
 \item Is there a systematic error?  (use at least two methods)
 \item Is there difference in precision?
\end{itemize}

%\answer{ 
%\begin{verbatim}
%K=read.table( "cvika_6_kalcit.dat", header=T, colClasses=c("numeric", "factor"))
%boxplot( weight ~ method, data=K)
%t.test( weight ~ method, data=K)
%var.test( weight ~ method, data=K)
%\end{verbatim}
%}


\subsection{ANOVA}
\begin{itemize}
 \item Get help for 'InsectSprays'.
 \item Display boxplots grouped by factors.
 \item Hypothesis: mean value of differs between groups.
 \item Get help and try to use: 'oneway.test' (allow different variance), 'aov' (assume balanced samples), 'lm' (general).
 
       
 \item Perform post-hoc analysis using 'TukeyHSD' and 'pairwise.t.test' with 'p.adjust' parameter.
\end{itemize}


\subsection{Corellation (A15)}
download datafile: \verb'http://atrey.karlin.mff.cuni.cz/~morf/vyuka/pas/cvika/cvika_6_asfalt.dat'
Melting point of asphalt and penetration into asphalt at 25\textdegree C

\begin{itemize}
 \item Are they independent? We assume 2D normal distribution.

 \item plot data to "see" dependency
 \item use various methods of cor.test
\end{itemize}
%ad 2) 
%cor.test( ~ temp + penetration, data=A)


\subsection{Queue statistics (A01)}

\begin{tabular}{lllllllll}
in queue        : &0  &1   &2   &3  &4  &5  &6  &7 and more\\
frequency       : &65 &121 &144 &99 &41 &15 &10 &5
\end{tabular}
Test that data are comming from Poisson distribution (Goodness of fit).



%lambda=(t(b_freq) %*% c(0:7))/sum(b_freq)
%func <- function(l) { pp=dpois(0:6,l); pp[8]=1-sum(pp); x=chisq.test(b_freq, p=pp); return <- x$p.value }
%y=lapply(x,func)
%plot(x,y)

%func <- function(l) { pp=dpois(0:6,l); pp[8]=1-sum(pp); x=chisq.test(b_freq, p=pp); return <- 1-x$p.value }
%optimise(func, interval=c( lambda-0.2, lambda+0.2) )

\noindent
\pagebreak

%%%%%%%%%%%%%%%%%%%%%%%%%%%%%%%%%%%%%%%%%%%%%%%%%%%%%%%%%%%%%%%%%%%%%%%%%%%%%%%%%%%%%%%%%%%%%%%%%%%%%%%

\section{Exercise 7}
% Regrese, ANOVA, kontingencni tabulky

%4) example(InsectSpray)
%
%Regression:
%  
%Z02     kvadraticka
%Z03     exponencialni 
%
%In R:
%
%plot(age,height)
%> res=lm(height~age)
%abline(res)
%summary(res) # tests about coefficiants
%coef(res)
%resid(res)
%fitted(res)
%? extrapolace do dalsich bodu
%predict(res, data.frame( old_Y = new_Y), level = 0.9,   interval="confidence")
%
%layout(matrix(1:4,2,2))
%plot(res)
%
%00 - residuals vs. fitted values (should be randomly around zero)
%10 - QQplot of residuals
%01 - sqrt(rel. err.) vs. fitted  (should not be trend
%
%plot(cooks.distance(lm.out))
%rlm() # from library(MASS)
%
%lines(lowess(Hwt ~ Bwt), col="red")
%scatter.smooth(Hwt ~ Bwt)
%









 






% 
% Rozdil mezi pouzitim T testu a Chi2 testu pro porovnani dvou vyberu (leva vs. prava pukla panve)
% 
% T - test umoznuje testovat shodu strednich hodnot (i pri velkych rozptylech pokud je velke i n)
% Chi2 - testuje shodu rozdeleni?
%
% ? pri interpolaci muze byt problem s nezavislosti nahodnych velicin !!
% skrtat ty nejblizsi dvojice, ktere se prekryvaji
% 
% Otazka: X,Y,Z jsou nezavisle, jsou nezavisle X-Z a Y-Z, nejsou staci uvazovat alternativni X,Y=0,1; Z=0,2
%


\end{document}

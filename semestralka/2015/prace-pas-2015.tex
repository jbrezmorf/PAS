\documentclass[a4paper,12pt]{article}
%\usepackage[active]{srcltx}
\usepackage[czech]{babel}
\usepackage[utf8]{inputenc}

\usepackage{amsmath}
\usepackage{amsfonts}
\usepackage{amssymb}
\usepackage{amsthm}
%
\newtheorem{theorem}{Věta}[section]
\newtheorem{proposition}[theorem]{Tvrzení}
\newtheorem{definition}[theorem]{Definice}
\newtheorem{remark}[theorem]{Poznámka}
\newtheorem{lemma}[theorem]{Lemma}
\newtheorem{corollary}[theorem]{Důsledek}
\newtheorem{exercise}[theorem]{Cvičení}

%\numberwithin{equation}{document}
%
\def\div{{\rm div}}
\def\Lapl{\Delta}
\def\grad{\nabla}
\def\supp{{\rm supp}}
\def\dist{{\rm dist}}
%\def\chset{\mathbbm{1}}
\def\chset{1}
%
\def\Tr{{\rm Tr}}
\def\to{\rightarrow}
\def\weakto{\rightharpoonup}
\def\imbed{\hookrightarrow}
\def\cimbed{\subset\subset}
\def\range{{\mathcal R}}
\def\leprox{\lesssim}
\def\argdot{{\hspace{0.18em}\cdot\hspace{0.18em}}}
\def\Distr{{\mathcal D}}
\def\calK{{\mathcal K}}
\def\FromTo{|\rightarrow}
\def\convol{\star}
\def\impl{\Rightarrow}
\DeclareMathOperator*{\esslim}{esslim}
\DeclareMathOperator*{\esssup}{ess\,supp}
\DeclareMathOperator{\ess}{ess}
\DeclareMathOperator{\osc}{osc}
\DeclareMathOperator{\curl}{curl}
%
%\def\Ess{{\rm ess}}
%\def\Exp{{\rm exp}}
%\def\Implies{\Longrightarrow}
%\def\Equiv{\Longleftrightarrow}
% ****************************************** GENERAL MATH NOTATION
\def\Real{{\rm\bf R}}
\def\Rd{{{\rm\bf R}^{\rm 3}}}
\def\RN{{{\rm\bf R}^N}}
\def\D{{\mathbb D}}
\def\Nnum{{\mathbb N}}
\def\Measures{{\mathcal M}}
\def\d{\,{\rm d}}               % differential
\def\sdodt{\genfrac{}{}{}{1}{\rm d}{{\rm d}t}}
\def\dodt{\genfrac{}{}{}{}{\rm d}{{\rm d}t}}
\def\dx{\d x}
\def\Lin{\mathit{Lin}}
%
\def\vc#1{\mathbf{\boldsymbol{#1}}}     % vector
\def\tn#1{{\mathbb{#1}}}    % tensor
\def\abs#1{\lvert#1\rvert}
\def\Abs#1{\bigl\lvert#1\bigr\rvert}
\def\bigabs#1{\bigl\lvert#1\bigr\rvert}
\def\Bigabs#1{\Big\lvert#1\Big\rvert}
\def\ABS#1{\left\lvert#1\right\rvert}
\def\norm#1{\bigl\Vert#1\bigr\Vert} %norm
\def\close#1{\overline{#1}}
\def\inter#1{#1^\circ}
\def\eqdef{\mathrel{\mathop:}=}     % defining equivalence
\def\where{\,|\,}                    % "where" separator in set's defs
\def\timeD#1{\dot{\overline{{#1}}}}
%
% ******************************************* USEFULL MACROS
\def\RomanEnum{\renewcommand{\labelenumi}{\rm (\roman{enumi})}}   % enumerate by roman numbers
\def\rf#1{(\ref{#1})}                                             % ref. shortcut
\def\prtl{\partial}                                        % partial deriv.
\def\Names#1{{\scshape #1}}
\def\rem#1{{\parskip=0cm\par!! {\sl\small #1} !!}}
\def\reseni#1{\par{\bf Řešení:}#1}
%
%
% ******************************************* DOCUMENT NOTATIONS
% document specific
%***************************************************************************
%
\addtolength{\textwidth}{2cm}
\addtolength{\vsize}{2cm}
\addtolength{\topmargin}{-1cm}
\addtolength{\hoffset}{-1cm}

\begin{document}
\parskip=2ex
\parindent=0pt
\pagestyle{empty}
 \section*{Seminární práce z PAS 2015}

 Následující úlohy vypracujte samostatně, vysvětlete jaký postup výpočtu použijete a proč, 
 uveďte příkazy použitého software, kterými jste výpočet realizovali a klíčové mezivýsledky. 
 U testů interpretujte obdržený výsledek.


\begin{enumerate}
%%%%%%%%%%%%%%%%%%%%%%%%%%%%%%%%%%%%%%%%%%%%%%%%%%%%%%%%%%%%%%%%%%%%%%%%%%%%%%5
  \item 
Firma na těžbu zlata chce porovnat výtěžnost dvou ložisek. U prvního byly naměřeny následující hodnoty 
na jednotlivých vzorcích: 3,78 8,25 12,05 11,29 11,1 9,5 10,61 11,44

U druhého ložiska byly naměřeny hodnoty:
10,91 15,63 6,24 10,69 13,14 10,41 15,39 13,91 5,83 14,86
9,35 16,01 15,52 11,8

Proveďte exploratorní analýzu pro oba výběry: vytvořte graf empirických distribučních funkcí pro oba výběry, porovnání krabicových grafů pro oba výběry,
pro každý výběr spočítejte: sřední hodnotu, rozptyl, šikmost, špičatost.

Za předpokladu shody rozptylů otestujte čistým testem významnosti, zda má jedno z ložisek statisticky významě vyšší výtěžnost
než druhé. 
%%%%%%%%%%%%%%%%%%%%%%%%%%%%%%%%%%%%%%%%%%%%%%%%%%%%%%%%%%%%%%%%%%%%%%%%%%%%%%5
  \item Při výrobě plastových výlisků G796 je zmetkovitost 4\%. Výrobce aplikuje při výstupu zkoušku, kterou výrobek projde s pravděpodobností 97\%,
  pokud je bezvadný a s provděpodobností 7\%, pokud je vadný.
  Vypočtěte kolik reklamací může výrobce čekat v sérii 1000 výrobků, pokud je výrobek expedován 
  a) po projití jednou výstupní zkouškou b) po projití dvěma nezávislými výstupními zkouškami.


  %%%%%%%%%%%%%%%%%%%%%%%%%%%%%%%%%%%%%%%%%%%%%%%%%%%%%%%%%%%%%%%%%%%%%%%%%%%%%%5
  \item  Při odečtu na stupnici měřícího přístroje se poslední číslice jen odhaduje a 
  teoreticky jsou všechny číslice stejně pravděpodobné, v praxi však mohou pozorovatelé 
  dávat některým hodnotám přednost. V tabulce je uvedeno 300 výsledků při odečtu poslední číslice. 
  Otestujte na hladině významnosti 0.05 zda jsou data zatížena systematickou chybou, nebo zda 
  data odpovídají rovnoměrnému rozdělení. 

    \begin{center}
    \begin{tabular}{lllllllllll}
    číslice &0 &1 &2 &3 &4 &5 &6 &7 &8 &9\\
    četnost &45 &26 &25 &27 &22 &32 &21 &26 &40 &34 
    \end{tabular}
    \end{center}


  %%%%%%%%%%%%%%%%%%%%%%%%%%%%%%%%%%%%%%%%%%%%%%%%%%%%%%%%%%%%%%%%%%%%%%%%%%%%%%5
\item
      Při měření závislosti hustoty vody $\rho$ na její teplotě $t$ byly naměřeny hodnoty (v kg/l):
\begin{center}
      \begin{tabular}{llllllllllll}
       t & 0 & 10 & 20 & 30 & 40 & 50 & 60 & 70 & 80 & 90 & 100 \\
       rho & 1,000 & 1,000 & 0,997 & 0,996 & 0,993 & 0,987 & 0.983 & 0.978 & 0.973 & 0.964 & 0.958
      \end{tabular}
\end{center}
      Odhadněte koeficienty kvadratického modelu, spočtěte reziduální rozptyl a testujte hypotézu $H_0$, že závislost je pouze lineární. 


  %%%%%%%%%%%%%%%%%%%%%%%%%%%%%%%%%%%%%%%%%%%%%%%%%%%%%%%%%%%%%%%%%%%%%%%%%%%%%%5
      

\end{enumerate}


\end{document}

\begin{enumerate}

%%%%%%%%%%%%%%%%%%% Probalility theory %%%%%%%%%%%%%%%%%%%%%%%%%%%%%%%%%%%%%%%%%%%
    \item Určete nejmenší počet hodů hrací kostkou $n(p)$ tak, aby pravděpodobnost padnutí alespoň jedné
        "šestky" byla minimálně $p$. Konkrétně určete $n(p)$ pro $p=0.9$ a pro $p=0.99$. Určete pravděpodobnost, 
        že mi první "šestka" padne v pátém hodu kostkou.

        %%%%%%%%%%%%%%%%%%%%%%%%%%%%%%%%%%%%%%%%%%%%%%%%%%%%%%%%%%%%%%%%%%%%%%%%%%%%%%5

    \item V pokeru se rozdává $5$ karet z $52$. Kolik je takových rozdání? 
  Určete pravděpodobnosti rozdání následujících karetních kombinací: 
  a) dvojice - dvě karty stejné hodnoty 
  b) trojice - tři karty stejné hodnoty 
  c) full house - trojice + dvojice 
  d) čistá postupka - postupka jedné barvy, přičemž eso může sloužit též jako 1.

%%%%%%%%%%%%%%%%%%%%%%%%%%%%%%%%%%%%%%%%%%%%%%%%%%%%%%%%%%%%%%%%%%%%%%%%%%%%%%5

  \item Při výrobě plastových výlisků G796 je zmetkovitost 4\%. Výrobce aplikuje při výstupu zkoušku, kterou výrobek projde s pravděpodobností 97\%,
  pokud je bezvadný a s provděpodobností 7\%, pokud je vadný.
  Vypočtěte kolik reklamací může výrobce čekat v sérii 1000 výrobků, pokud je výrobek expedován 
  a) po projití jednou výstupní zkouškou b) po projití dvěma nezávislými výstupními zkouškami.

  %%%%%%%%%%%%%%%%%%%%%%%%%%%%%%%%%%%%%%%%%%%%%%%%%%%%%%%%%%%%%%%%%%%%%%%%%%%%%%5

    \item Výrobce předává zákazníkovi dodávku 200 výrobků z nichž 25 je vadných. 
  Zákazník provádí kontrolu testem deseti náhodně vybraných výrobků. 
  A přijme dodávku pokud je max. jeden výrobek vadný. Jaká je pravděpodobnost přijetí dodávky? 

  Z 200 dodaných výrobků je $M$ vadných. Zákazník $M$ nezná a provádí kontrolu deseti výrobků. Určete pro jaké $M$
  je největší pravděpodobnost, že při kontrole budou 2 výrobky vadné. Určete i uvedenou pravděpodobnost.

  Pomocí Bayesovy věty nalezněte nejpravděpodobnější hodnotu $M$ za podmínky, že při kontrole budou 2 výrobky vadné.
  Předpokládejte, že pravděpodobnost výroby jednoho vadného výrobku je 10\%.

  %%%%%%%%%%%%%%%%%%%%%%%%%%%%%%%%%%%%%%%%%%%%%%%%%%%%%%%%%%%%%%%%%%%%%%%%%%%%%%5

  \item Součástky A, B, C jsou zapojeny sériově, tj. při poruše kterékoliv z nich dojde k poruše celého systému. Doba do poruchy
      součástek má exponenciální rozdělení a poruchy na součástkéch jsou nezávislé. Střední doby do poruchy jsou pro součástku A 1000 h, pro B 2000 h a pro C 1200 h.
      Vypočtěte střední dobu do poruchy celého systému. Pro každou součástku stanovte pravděpodobnost, že bude fungovat alespoň 500 h. 
      Najděte distribuční funkci doby do poruchy celého systému a určete pravděpodobnost, že celý systém bude v provozuschopném stavu v čase t=200 h. 

      %%%%%%%%%%%%%%%%%%%%%%%%%%%%%%%%%%%%%%%%%%%%%%%%%%%%%%%%%%%%%%%%%%%%%%%%%%%%%%5

  \item Určete nejmenší počet hodů hrací kostkou $n(p)$ tak, aby pravděpodobnost padnutí alespoň jedné
        "šestky" byla minimálně $p$. Konkrétně určete $n(p)$ pro $p=0,9$ a pro $p=0.99$. Určete pravděpodobnost, 
        že mi první "šestka" padne v pátém hodu kostkou.

      

%%%%%%%%%%%%%%%%%%%%%% Exploratory statistics %%%%%%%%%%%%%%%%%%%%%%%%%%%%%%%%%%%%%%%
  \item Při dopravním průzkumu byla sledována vytíženost vjezdu do křižovatky.
  Student, provádějící průzkum, si vždy při rozsvícení zeleného světla zapsal počet aut, čekajících 
  ve frontě u semaforu (náhodná veličina $X$). Při rozsvícené červené byla fronta vždy prázdná. Jeho zapsané výsledky jsou:

4, 6, 3, 3, 6, 4, 9, 7, 5, 4, 2, 7, 5, 3, 4, 4, 7, 4, 1, 6, 4, 5, 8, 3, 5,
7, 4, 8, 5, 7, 2, 7, 3, 4, 3, 3, 3, 3, 5, 8, 4, 2, 9, 5, 1, 10, 8, 0, 3, 4,
5, 4, 5, 4, 5, 3, 5, 6, 3, 2, 11, 10, 4, 5, 4, 5, 9, 3, 5, 7, 3, 3, 2, 7, 3,
7, 3, 4, 5, 5, 3, 1, 5, 2, 7, 4, 3, 7, 3, 6, 5, 3, 2, 4, 2, 4, 6, 3, 6, 3

  Spočtěte: průměr, výběrovou odchylku, medián, a IQR. Vytvořte tabulku četností, nakreslete histogram, krabicový graf a empirickou distribucni funkci. 
  Porovnejte data s normálním a Poisonovým rozdělením pomocí Q-Q grafu, 
  pro parametry rozdělení použijte bodové odhady (výběrový průměr a odchylku). Proč by náhodná veličina $X$ mohla mít Poissonovo rozdělení 
  (bez ohledu na získaná data)?

  %%%%%%%%%%%%%%%%%%%%%%%%%%%%%%%%%%%%%%%%%%%%%%%%%%%%%%%%%%%%%%%%%%%%%%%%%%%%%%5

  \item 
  Ústav pro zjišťování životnosti výrobků dostal zakázku na testování životnosti 
žárovek, kterých testoval 50 kusů. Testované žárovky byly sledovány až do své 
poruchy. Byly naměřeny následující doby do poruchy (v hodinách): 


932, 1130, 974, 924, 1234, 1246, 852, 841, 1097, 1056, 1006, 796, 1018, 973, 1091, 769, 1022, 952, 832, 962, 1074, 1042, 1071, 1215, 904, 915, 963, 1095, 1099, 1014, 878, 933, 1116, 1008, 945, 1166, 720, 1137, 970, 857, 879, 1068, 1047, 978, 922, 372, 897, 1153, 1366, 1052


  Vypočtěte: průměr, směrodatnou odchylku, medián, a IQR. 
  Identifikujte odlehlá pozorování dvěma různými metodami. Vypočtěte průměr a směrodatnou odchylku pro filtrovaná data.
  Zorazte histogram, krabicový graf a empirickou distribucni funkci. 
  
  Porovnejte data s normálním pomocí Q-Q grafu.
  Za předpokloadu normality rozdělení dat určete pravděpodobnost, že 
  žárovka vydrží v provozu alespoň 750 hodin.

  \item 
Firma na těžbu zlata chce porovnat výtěžnost dvou ložisek. U prvního byly naměřeny následující hodnoty 
na jednotlivých vzorcích: 3,78 8,25 12,05 11,29 11,1 9,5 10,61 11,44

U druhého ložiska byly naměřeny hodnoty:
10,91 15,63 6,24 10,69 13,14 10,41 15,39 13,91 5,83 14,86
9,35 16,01 15,52 11,8

Proveďte exploratorní analýzu pro oba výběry: vytvořte graf empirických distribučních funkcí pro oba výběry, porovnání krabicových grafů pro oba výběry,
pro každý výběr spočítejte: sřední hodnotu, rozptyl, šikmost, špičatost.

Za předpokladu shody rozptylů otestujte čistým testem významnosti, zda má jedno z ložisek statisticky významě vyšší výtěžnost
než druhé. 




%%%%%%%%%%%%%%%%%%%%%% Basic tests (normal distribution) %%%%%%%%%%%%%%%%%%%%%%%%%%%%

\item 
Trajekt pro převoz má užitečnou nosnost $5000$ kg. Sledováním bylo zjištěno, že hmotnost
převážených cestujících je náhodná veličina se střední hodnotou $70$ kg a střední směrodatnou
odchylkou $20$ kg. Jakou pravděpodobnost přetížení trajektu můžeme očekávat při převozu $65$
cestujících a kolik můžeme maximálně cestujících převážet, aby pravděpodobnost přetížení
byla maximálně $0.001$.

%%%%%%%%%%%%%%%%%%%%%%%%%%%%%%%%%%%%%%%%%%%%%%%%%%%%%%%%%%%%%%%%%%%%%%%%%%%%%%5

  \item Bylo vyšetřeno $n_1 = 1600$ dvanáctiletých dětí ve městě, kde byla pitná voda fluorována
 a $n_2 = 900$ dvanáctiletých dětí v blízkém městě, kde fluorována nebyla.
U každého dítěte byl zjištěn index kazivosti chrupu. 
Pro první skupinu byla vypočtena průměrná kazivost $3.288$ s výběrovou odchylkou $0.649$ pro 
druhou skupinu byl průměr $3.314$ a odchylka $0.592$. Umožňují tato data spolehlivě prohlásit, že 
fluorování vody snižuje kazivost? Použijte čistý test významnosti a pečlivě formulujte předpoklady testu.

%%%%%%%%%%%%%%%%%%%%%%%%%%%%%%%%%%%%%%%%%%%%%%%%%%%%%%%%%%%%%%%%%%%%%%%%%%%%%%5

  \item Při měření mechanického napětí pomocí lepených tenzometrů může časem dojít ke porušení lepeného spoje.
  Tenzometr by pak vykazoval menší hodnoty. V následující tabulce jsou data z tenzometru nově nalepeného (A) a ze 
  stejného tenzometru po půl roce provozu (B). Data pocházejí ze stejné série mechanického zatížení (load). Pomocí čistého testu 
  významnosti otestujte, zda starý tenzometr vykazuje nižší hodnoty než nový.
  
  load: 100, 110, 120, 210, 220, 230, 310, 320, 330\\
  A: 12.36, 13.60, 14.79, 25.92, 27.29, 28.19, 37.92, 39.95, 39.92\\
  B: 12.19, 13.44, 14.60, 25.53, 26.36, 27.94, 37.75, 38.61, 40.60

  %%%%%%%%%%%%%%%%%%%%% Nonparametis tests, Chi square %%%%%%%%%%%%%%%%%%%%%%%%%%%%%%%%%%%%%%%%
  \item Při dopravním průzkumu byla sledována vytíženost vjezdu do křižovatky.
  Student, provádějící průzkum, si vždy při rozsvícení zeleného světla zapsal počet aut, čekajících 
  ve frontě u semaforu (náhodná veličina $X$). Při rozsvícené červené byla fronta vždy prázdná. Jeho zapsané výsledky jsou:

4, 6, 3, 3, 6, 4, 9, 7, 5, 4, 2, 7, 5, 3, 4, 4, 7, 4, 1, 6, 4, 5, 8, 3, 5,
7, 4, 8, 5, 7, 2, 7, 3, 4, 3, 3, 3, 3, 5, 8, 4, 2, 9, 5, 1, 10, 8, 0, 3, 4,
5, 4, 5, 4, 5, 3, 5, 6, 3, 2, 11, 10, 4, 5, 4, 5, 9, 3, 5, 7, 3, 3, 2, 7, 3,
7, 3, 4, 5, 5, 3, 1, 5, 2, 7, 4, 3, 7, 3, 6, 5, 3, 2, 4, 2, 4, 6, 3, 6, 3
  
  Pomocí znaménkového testu ověřte hypotézu, že v polovině případů je počet aut ve frontě menší než $4$.

  %%%%%%%%%%%%%%%%%%%%%%%%%%%%%%%%%%%%%%%%%%%%%%%%%%%%%%%%%%%%%%%%%%%%%%%%%%%%%%5

  \item  Při odečtu na stupnici měřícího přístroje se poslední číslice jen odhaduje a 
  teoreticky jsou všechny číslice stejně pravděpodobné, v praxi však mohou pozorovatelé 
  dávat některým hodnotám přednost. V tabulce je uvedeno třista výsledků při odečtu poslední číslice. 
  Otestujte na hladině významnosti 0.05 zda jsou data zatížena systematickou chybou, nebo zda 
  data odpovídají rovnoměrnému rozdělení. Použijte test dobré shody nebo Kolmogorovův test.

    \begin{center}
    \begin{tabular}{lllllllllll}
    číslice &0 &1 &2 &3 &4 &5 &6 &7 &8 &9\\
    četnost &45 &26 &25 &27 &22 &32 &21 &26 &40 &34 
    \end{tabular}
    \end{center}

    %%%%%%%%%%%%%%%%%%%%%%%%%%%%%%%%%%%%%%%%%%%%%%%%%%%%%%%%%%%%%%%%%%%%%%%%%%%%%%5

  \item Ve 150-ti minutových intervalech byl určen počet spojených hovorů podle následující tabulky četností.
        Určete vhodný typ teoretického rozdělení, odhadněte jeho parametry a testujte shodu s tímto rozdělením 
        testem dobré shody nebo Kolmogorovovým testem. Proveďte čistý test významnosti.
    \begin{center}
    \begin{tabular}{lllllll}
    Počet vad           &0& 1& 2& 3& 4& 5& 6\\
    Počet intervalů     &41& 47& 36& 18& 5& 2& 1
    \end{tabular}
    \end{center}



%%%%%%%%%%%%%%%%%%%%% ANOVA, regression %%%%%%%%%%%%%%%%%%%%%%%%%%%%%%%%
  \item Pacienti podstupující bypass bylli náhodně rozděleni do tří skupin podle druhu ventilace.
  \begin{enumerate}
    \item Pacienti dýchají 24 hodin směs 50\% $N_2O$ a 50\% $O_2$.
    \item Pacienti dýchají směs 50\% $N_2O$ a 50\% $O_2$ pouze během operace.
    \item Pacienti dýchají 24 hodin směs 50\% $N_2$ a 50\% $O_2$ (vzduch s větším obsahem kyslíku). 
  \end{enumerate}
  Každému paciantovi byla následně změřena koncentrace vitaminu B v červených krvinkách (prorůstový faktor). Výsledky jsou:
  
  skupina 1: 243, 251, 275, 291, 347, 354, 380, 392, 309\\
  skupina 2: 206, 210, 226, 249, 255, 273, 285, 295\\
  skupina 3: 241, 258, 270, 293, 328
  
  Vytvořte bodový graf pro jednotlivé skupiny. Vytvořte předběžný odhad výsledků ANOVY.
  Proveďte analýzu rozptylu s jednoduchým tříděním a post-hoc analýzu 
  v případě významného rozdílu mezi skupinami.  Diskutujte splnění předpokladů testu.

  %%%%%%%%%%%%%%%%%%%%%%%%%%%%%%%%%%%%%%%%%%%%%%%%%%%%%%%%%%%%%%%%%%%%%%%%%%%%%%5

  \item Byla zkoumána atraktivita čtyř barev pro určitý druh hmyzu. Pro každou barvu bylo provedeno $7$ pokusů.
        Pro žlutou byl průmerný počet chyceného hmyzu $47.167$ se standardní odchylkou $6.795$. Pro bílou byl průměr 
        $15.667$, odchylka $3.327$. Pro zelenou $31.5$ a $9.915$, pro modrou $14.833$ a $5.345$. Proveďte analýzu rozptylu.

        %%%%%%%%%%%%%%%%%%%%%%%%%%%%%%%%%%%%%%%%%%%%%%%%%%%%%%%%%%%%%%%%%%%%%%%%%%%%%%5

  \item Barmanka ``U zelené sedmy'' si přividělává, aby ušetřila na ``zápisné''. Pro mentalní osvěžení mezi točením půllitrů
  sleduje kolik chodí zákazníků z různých fakult. Během posledních dní si poznamenala následující počty

    \begin{align*}
    \text{Strojní: } 21, 16, 17, 18 \\
    \text{Mechatronika: } 6, 10, 13, 13, 4, 8\\
    \text{Textilní: } 8, 6, 4, 5, 2\\
    \end{align*}

  Zjistěte zda se návštěvnost mění v závislosti na fakultě, a které rozdíly jsou staticticky významné, použijte hladinu významnosti (10\%).
  Jsou splněny předpoklady testů?

  %%%%%%%%%%%%%%%%%%%%%%%%%%%%%%%%%%%%%%%%%%%%%%%%%%%%%%%%%%%%%%%%%%%%%%%%%%%%%%5

  
\item
      Při měření závislosti hustoty vody $\rho$ na její teplotě $t$ byly naměřeny hodnoty (v kg/l):
\begin{center}
      \begin{tabular}{llllllllllll}
       t & 0 & 10 & 20 & 30 & 40 & 50 & 60 & 70 & 80 & 90 & 100 \\
       rho & 1,000 & 1,000 & 0,997 & 0,996 & 0,993 & 0,987 & 0.983 & 0.978 & 0.973 & 0.964 & 0.958
      \end{tabular}
\end{center}
      Odhadněte koeficienty kvadratického modelu, spočtěte reziduální rozptyl a testujte hypotézu $H_0$, že závislost je pouze lineární. 

\end{enumerate}

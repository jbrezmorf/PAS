\documentclass[a4paper,12pt]{article}
%\usepackage[active]{srcltx}
\usepackage[czech]{babel}
\usepackage[utf8]{inputenc}

\usepackage{amsmath}
\usepackage{amsfonts}
\usepackage{amssymb}
\usepackage{amsthm}
%
\newtheorem{theorem}{Věta}[section]
\newtheorem{proposition}[theorem]{Tvrzení}
\newtheorem{definition}[theorem]{Definice}
\newtheorem{remark}[theorem]{Poznámka}
\newtheorem{lemma}[theorem]{Lemma}
\newtheorem{corollary}[theorem]{Důsledek}
\newtheorem{exercise}[theorem]{Cvičení}

%\numberwithin{equation}{document}
%
\def\div{{\rm div}}
\def\Lapl{\Delta}
\def\grad{\nabla}
\def\supp{{\rm supp}}
\def\dist{{\rm dist}}
%\def\chset{\mathbbm{1}}
\def\chset{1}
%
\def\Tr{{\rm Tr}}
\def\to{\rightarrow}
\def\weakto{\rightharpoonup}
\def\imbed{\hookrightarrow}
\def\cimbed{\subset\subset}
\def\range{{\mathcal R}}
\def\leprox{\lesssim}
\def\argdot{{\hspace{0.18em}\cdot\hspace{0.18em}}}
\def\Distr{{\mathcal D}}
\def\calK{{\mathcal K}}
\def\FromTo{|\rightarrow}
\def\convol{\star}
\def\impl{\Rightarrow}
\DeclareMathOperator*{\esslim}{esslim}
\DeclareMathOperator*{\esssup}{ess\,supp}
\DeclareMathOperator{\ess}{ess}
\DeclareMathOperator{\osc}{osc}
\DeclareMathOperator{\curl}{curl}
%
%\def\Ess{{\rm ess}}
%\def\Exp{{\rm exp}}
%\def\Implies{\Longrightarrow}
%\def\Equiv{\Longleftrightarrow}
% ****************************************** GENERAL MATH NOTATION
\def\Real{{\rm\bf R}}
\def\Rd{{{\rm\bf R}^{\rm 3}}}
\def\RN{{{\rm\bf R}^N}}
\def\D{{\mathbb D}}
\def\Nnum{{\mathbb N}}
\def\Measures{{\mathcal M}}
\def\d{\,{\rm d}}               % differential
\def\sdodt{\genfrac{}{}{}{1}{\rm d}{{\rm d}t}}
\def\dodt{\genfrac{}{}{}{}{\rm d}{{\rm d}t}}
\def\dx{\d x}
\def\Lin{\mathit{Lin}}
%
\def\vc#1{\mathbf{\boldsymbol{#1}}}     % vector
\def\tn#1{{\mathbb{#1}}}    % tensor
\def\abs#1{\lvert#1\rvert}
\def\Abs#1{\bigl\lvert#1\bigr\rvert}
\def\bigabs#1{\bigl\lvert#1\bigr\rvert}
\def\Bigabs#1{\Big\lvert#1\Big\rvert}
\def\ABS#1{\left\lvert#1\right\rvert}
\def\norm#1{\bigl\Vert#1\bigr\Vert} %norm
\def\close#1{\overline{#1}}
\def\inter#1{#1^\circ}
\def\eqdef{\mathrel{\mathop:}=}     % defining equivalence
\def\where{\,|\,}                    % "where" separator in set's defs
\def\timeD#1{\dot{\overline{{#1}}}}
%
% ******************************************* USEFULL MACROS
\def\RomanEnum{\renewcommand{\labelenumi}{\rm (\roman{enumi})}}   % enumerate by roman numbers
\def\rf#1{(\ref{#1})}                                             % ref. shortcut
\def\prtl{\partial}                                        % partial deriv.
\def\Names#1{{\scshape #1}}
\def\rem#1{{\parskip=0cm\par!! {\sl\small #1} !!}}
\def\reseni#1{\par{\bf Řešení:}#1}
%
%
% ******************************************* DOCUMENT NOTATIONS
% document specific
%***************************************************************************
%
%\addtolength{\textwidth}{2cm}
%\addtolength{\vsize}{2cm}
%\addtolength{\topmargin}{-1cm}
%\addtolength{\hoffset}{-1cm}
\begin{document}
\parskip=2ex
\parindent=0pt
\pagestyle{empty}
 \section*{Seminární práce z PAS 2013}

 Následující úlohy vypracujte samostatně, vysvětlete jaký postup výpočtu použijete a proč, uveďte příkazy software R, kterými jste výpočet realizovali.


 \begin{enumerate}
  \item Při výrobě plastových výlisků G796 je zmetkovitost 4\%. Výrobce aplikuje při výstupu zkoušku, kterou výrobek projde s pravděpodobností 97\%,
  pokud je bezvadný a s provděpodobností 7\%, pokud je vadný.
  Vypočtěte kolik reklamací může výrobce čekat v sérii 1000 výrobků, pokud je výrobek expedován 
  a) po projití jednou výstupní zkouškou b) po projití dvěma nezávislými výstupními zkouškami.

  \item Při dopravním průzkumu byla sledována vytíženost vjezdu do křižovatky.
  Student, provádějící průzkum, si vždy při rozsvícení zeleného světla zapsal počet aut, čekajících 
  ve frontě u semaforu (náhodná veličina $X$). Při rozsvícené červené byla fronta vždy prázdná. Jeho zapsané výsledky jsou:

4, 6, 3, 3, 6, 4, 9, 7, 5, 4, 2, 7, 5, 3, 4, 4, 7, 4, 1, 6, 4, 5, 8, 3, 5,
7, 4, 8, 5, 7, 2, 7, 3, 4, 3, 3, 3, 3, 5, 8, 4, 2, 9, 5, 1, 10, 8, 0, 3, 4,
5, 4, 5, 4, 5, 3, 5, 6, 3, 2, 11, 10, 4, 5, 4, 5, 9, 3, 5, 7, 3, 3, 2, 7, 3,
7, 3, 4, 5, 5, 3, 1, 5, 2, 7, 4, 3, 7, 3, 6, 5, 3, 2, 4, 2, 4, 6, 3, 6, 3


  Spočtěte: průměr, výběrovou odchylku, medián, a IQR. Vytvořte tabulku četností, nakreslete histogram, krabicový graf a empirickou distribucni funkci. 
  Porovnejte data s normálním a Poisonovým rozdělením pomocí Q-Q grafu, 
  pro parametry rozdělení použijte bodové odhady (výběrový průměr a odchylku). Proč by náhodná veličina $X$ mohla mít Poissonovo rozdělení (bez ohledu na získaná data)?
  
  Pomocí znaménkového testu ověřte hypotézu, že v polovině případů je počet aut ve frontě menší než $4$.

  \item Bylo vyšetřeno $n_1 = 1600$ dvanáctiletých dětí ve městě, kde byla pitná voda fluorována
 a $n_2 = 900$ dvanáctiletých dětí v blízkém městě, kde fluorována nebyla.
U každého dítěte byl zjištěn index kazivosti chrupu. 
Pro první skupinu byla vypočtena průměrná kazivost $3.288$ s výběrovou odchylkou $0.649$ pro 
druhou skupinu byl průměr $3.314$ a odchylka $0.592$. Umožňují tato data spolehlivě prohlásit, že 
fluorování vody snižuje kazivost? Použijte čistý test významnosti a pečlivě formulujte předpoklady testu.

  
  \item  Při odečtu na stupnici měřícího přístroje se poslední číslice jen odhaduje a 
  teoreticky jsou všechny číslice stejně pravděpodobné, v praxi však mohou pozorovatelé 
  dávat některým hodnotám přednost. V tabulce je uvedeno třista výsledků při odečtu poslední číslice. 
  Otestujte na hladině významnosti 0.05 zda jsou data zatížena systematickou chybou, nebo zda 
  data odpovídají rovnoměrnému rozdělení. Použijte test dobré shody nebo Kolmogorovův test.

\begin{center}
\begin{tabular}{lllllllllll}
číslice &0 &1 &2 &3 &4 &5 &6 &7 &8 &9\\
četnost &45 &26 &25 &27 &22 &32 &21 &26 &40 &34 
\end{tabular}
\end{center}

 
  \item Pacienti podstupující bypass bylli náhodně rozděleni do tří skupin podle druhu ventilace.
  \begin{enumerate}
    \item Pacienti dýchají 24 hodin směs 50\% $N_2O$ a 50\% $O_2$.
    \item Pacienti dýchají směs 50\% $N_2O$ a 50\% $O_2$ pouze během operace.
    \item Pacienti dýchají 24 hodin směs 50\% $N_2$ a 50\% $O_2$ (vzduch s větším obsahem kyslíku). 
  \end{enumerate}
  Každému paciantovi byla následně změřena koncentrace vitaminu B v červených krvinkách (prorůstový faktor). Výsledky jsou:
  
  skupina 1: 243, 251, 275, 291, 347, 354, 380, 392, 309\\
  skupina 2: 206, 210, 226, 249, 255, 273, 285, 295\\
  skupina 3: 241, 258, 270, 293, 328
  
  Vytvořte bodový graf pro jednotlivé skupiny. Vytvořte předběžný odhad výsledků ANOVY.
  Proveďte analýzu rozptylu s jednoduchým tříděním a post-hoc analýzu 
  v případě významného rozdílu mezi skupinami.  Diskutujte splnění předpokladů testu.
\end{enumerate}


\end{document}

\begin{enumerate}

\item Součástky A, B, C jsou zapojeny sériově, tj. při poruše kterékoliv z nich dojde k poruše celého systému. Doba do poruchy
      součástek má exponenciální rozdělení a poruchy na součástkéch jsou nezávislé. Střední doby do poruchy jsou pro součástku A 1000 h, pro B 2000 h a pro C 1200 h.
      Vypočtěte střední dobu do poruchy celého systému. Pro každou součástku stanovte pravděpodobnost, že bude fungovat alespoň 500 h. Najděte distribuční funkci doby do poruchy celého systému a určete pravděpodobnost, že celý systém bude v provozuschopném stavu v čase t=200 h. 

  \item Firma na těžbu zlata chce porovnat výtěžnost dvou ložisek. U prvního byly naměřeny následující hodnoty na jednotlivých vzorcích: 3,78 8,25 12,05 11,29 11,1 9,5 10,61 11,44

U druhého ložiska byly naměřeny hodnoty:
10,91 15,63 6,24 10,69 13,14 10,41 15,39 13,91 5,83 14,86
9,35 16,01 15,52 11,8

Proveďte exploratorní analýzu pro oba výběry: vytvořte graf empirických distribučních funkcí pro oba výběry, spočítejte:
sřední hodnotu, rozptyl, šikmost, špičatost

Otestujte shodu rozptylů. 
Za předpokladu shody rozptylů otestujte na hladině významnosti $0,05$ zda jsou střední hodnoty výtěžnosti ložisek stejné.
\item
      Při měření závislosti hustoty vody $\rho$ na její teplotě $t$ byly naměřeny hodnoty (v kg/l):
\begin{center}
      \begin{tabular}{llllllllllll}
       t & 0 & 10 & 20 & 30 & 40 & 50 & 60 & 70 & 80 & 90 & 100 \\
       rho & 1,000 & 1,000 & 0,997 & 0,996 & 0,993 & 0,987 & 0.983 & 0.978 & 0.973 & 0.964 & 0.958
      \end{tabular}

\end{center}
      Odhadněte koeficienty kvadratického modelu, spočtěte reziduální rozptyl a testujte hypotézu $H_0$, že závislost je pouze lineární. 
  \end{enumerate}
